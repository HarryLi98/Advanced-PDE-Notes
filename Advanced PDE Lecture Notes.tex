\documentclass{report}

%%%%%%%%%%%%%%%%%%%%%%%%%%%%%%%%%
% PACKAGE IMPORTS
%%%%%%%%%%%%%%%%%%%%%%%%%%%%%%%%%


\usepackage[tmargin=2cm,rmargin=1in,lmargin=1in,margin=0.85in,bmargin=2cm,footskip=.2in]{geometry}
\usepackage{amsmath,amsfonts,amsthm,amssymb,mathtools}
\usepackage[varbb]{newpxmath}
\usepackage{xfrac}
\usepackage[makeroom]{cancel}
\usepackage{mathtools}
\usepackage{bookmark}
\usepackage{enumitem}
\usepackage{hyperref,theoremref}
\hypersetup{
	pdftitle={Assignment},
	colorlinks=true, linkcolor=doc!90,
	bookmarksnumbered=true,
	bookmarksopen=true
}
\usepackage[most,many,breakable]{tcolorbox}
\usepackage{xcolor}
\usepackage{varwidth}
\usepackage{varwidth}
\usepackage{etoolbox}
%\usepackage{authblk}
\usepackage{nameref}
\usepackage{multicol,array}
\usepackage{tikz-cd}
\usepackage[ruled,vlined,linesnumbered]{algorithm2e}
\usepackage{comment} % enables the use of multi-line comments (\ifx \fi) 
\usepackage{import}
\usepackage{xifthen}
\usepackage{pdfpages}
\usepackage{transparent}

\newcommand\mycommfont[1]{\footnotesize\ttfamily\textcolor{blue}{#1}}
\SetCommentSty{mycommfont}
\newcommand{\incfig}[1]{%
    \def\svgwidth{\columnwidth}
    \import{./figures/}{#1.pdf_tex}
}

\usepackage{tikzsymbols}
\renewcommand\qedsymbol{$\Laughey$}


%\usepackage{import}
%\usepackage{xifthen}
%\usepackage{pdfpages}
%\usepackage{transparent}


%%%%%%%%%%%%%%%%%%%%%%%%%%%%%%
% SELF MADE COLORS
%%%%%%%%%%%%%%%%%%%%%%%%%%%%%%



\definecolor{myg}{RGB}{56, 140, 70}
\definecolor{myb}{RGB}{45, 111, 177}
\definecolor{myr}{RGB}{199, 68, 64}
\definecolor{mytheorembg}{HTML}{F2F2F9}
\definecolor{mytheoremfr}{HTML}{00007B}
\definecolor{mylemmabg}{HTML}{FFFAF8}
\definecolor{mylemmafr}{HTML}{983b0f}
\definecolor{mypropbg}{HTML}{f2fbfc}
\definecolor{mypropfr}{HTML}{191971}
\definecolor{myexamplebg}{HTML}{F2FBF8}
\definecolor{myexamplefr}{HTML}{88D6D1}
\definecolor{myexampleti}{HTML}{2A7F7F}
\definecolor{mydefinitbg}{HTML}{E5E5FF}
\definecolor{mydefinitfr}{HTML}{3F3FA3}
\definecolor{notesgreen}{RGB}{0,162,0}
\definecolor{myp}{RGB}{197, 92, 212}
\definecolor{mygr}{HTML}{2C3338}
\definecolor{myred}{RGB}{127,0,0}
\definecolor{myyellow}{RGB}{169,121,69}
\definecolor{myexercisebg}{HTML}{F2FBF8}
\definecolor{myexercisefg}{HTML}{88D6D1}


%%%%%%%%%%%%%%%%%%%%%%%%%%%%
% TCOLORBOX SETUPS
%%%%%%%%%%%%%%%%%%%%%%%%%%%%

\setlength{\parindent}{1cm}
%================================
% THEOREM BOX
%================================

\tcbuselibrary{theorems,skins,hooks}
\newtcbtheorem[number within=section]{Theorem}{Theorem}
{%
	enhanced,
	breakable,
	colback = mytheorembg,
	frame hidden,
	boxrule = 0sp,
	borderline west = {2pt}{0pt}{mytheoremfr},
	sharp corners,
	detach title,
	before upper = \tcbtitle\par\smallskip,
	coltitle = mytheoremfr,
	fonttitle = \bfseries\sffamily,
	description font = \mdseries,
	separator sign none,
	segmentation style={solid, mytheoremfr},
}
{th}

\tcbuselibrary{theorems,skins,hooks}
\newtcbtheorem[number within=chapter]{theorem}{Theorem}
{%
	enhanced,
	breakable,
	colback = mytheorembg,
	frame hidden,
	boxrule = 0sp,
	borderline west = {2pt}{0pt}{mytheoremfr},
	sharp corners,
	detach title,
	before upper = \tcbtitle\par\smallskip,
	coltitle = mytheoremfr,
	fonttitle = \bfseries\sffamily,
	description font = \mdseries,
	separator sign none,
	segmentation style={solid, mytheoremfr},
}
{th}


\tcbuselibrary{theorems,skins,hooks}
\newtcolorbox{Theoremcon}
{%
	enhanced
	,breakable
	,colback = mytheorembg
	,frame hidden
	,boxrule = 0sp
	,borderline west = {2pt}{0pt}{mytheoremfr}
	,sharp corners
	,description font = \mdseries
	,separator sign none
}

%================================
% Corollery
%================================
\tcbuselibrary{theorems,skins,hooks}
\newtcbtheorem[number within=section]{Corollary}{Corollary}
{%
	enhanced
	,breakable
	,colback = myp!10
	,frame hidden
	,boxrule = 0sp
	,borderline west = {2pt}{0pt}{myp!85!black}
	,sharp corners
	,detach title
	,before upper = \tcbtitle\par\smallskip
	,coltitle = myp!85!black
	,fonttitle = \bfseries\sffamily
	,description font = \mdseries
	,separator sign none
	,segmentation style={solid, myp!85!black}
}
{th}
\tcbuselibrary{theorems,skins,hooks}
\newtcbtheorem[number within=chapter]{corollary}{Corollary}
{%
	enhanced
	,breakable
	,colback = myp!10
	,frame hidden
	,boxrule = 0sp
	,borderline west = {2pt}{0pt}{myp!85!black}
	,sharp corners
	,detach title
	,before upper = \tcbtitle\par\smallskip
	,coltitle = myp!85!black
	,fonttitle = \bfseries\sffamily
	,description font = \mdseries
	,separator sign none
	,segmentation style={solid, myp!85!black}
}
{th}


%================================
% lemma
%================================

\tcbuselibrary{theorems,skins,hooks}
\newtcbtheorem[number within=section]{lemma}{lemma}
{%
	enhanced,
	breakable,
	colback = mylemmabg,
	frame hidden,
	boxrule = 0sp,
	borderline west = {2pt}{0pt}{mylemmafr},
	sharp corners,
	detach title,
	before upper = \tcbtitle\par\smallskip,
	coltitle = mylemmafr,
	fonttitle = \bfseries\sffamily,
	description font = \mdseries,
	separator sign none,
	segmentation style={solid, mylemmafr},
}
{th}

%================================
% PROPOSITION
%================================

\tcbuselibrary{theorems,skins,hooks}
\newtcbtheorem[number within=section]{Prop}{Proposition}
{%
	enhanced,
	breakable,
	colback = mypropbg,
	frame hidden,
	boxrule = 0sp,
	borderline west = {2pt}{0pt}{mypropfr},
	sharp corners,
	detach title,
	before upper = \tcbtitle\par\smallskip,
	coltitle = mypropfr,
	fonttitle = \bfseries\sffamily,
	description font = \mdseries,
	separator sign none,
	segmentation style={solid, mypropfr},
}
{th}

\tcbuselibrary{theorems,skins,hooks}
\newtcbtheorem[number within=chapter]{prop}{Proposition}
{%
	enhanced,
	breakable,
	colback = mypropbg,
	frame hidden,
	boxrule = 0sp,
	borderline west = {2pt}{0pt}{mypropfr},
	sharp corners,
	detach title,
	before upper = \tcbtitle\par\smallskip,
	coltitle = mypropfr,
	fonttitle = \bfseries\sffamily,
	description font = \mdseries,
	separator sign none,
	segmentation style={solid, mypropfr},
}
{th}


%================================
% CLAIM
%================================

\tcbuselibrary{theorems,skins,hooks}
\newtcbtheorem[number within=section]{claim}{Claim}
{%
	enhanced
	,breakable
	,colback = myg!10
	,frame hidden
	,boxrule = 0sp
	,borderline west = {2pt}{0pt}{myg}
	,sharp corners
	,detach title
	,before upper = \tcbtitle\par\smallskip
	,coltitle = myg!85!black
	,fonttitle = \bfseries\sffamily
	,description font = \mdseries
	,separator sign none
	,segmentation style={solid, myg!85!black}
}
{th}



%================================
% Exercise
%================================

\tcbuselibrary{theorems,skins,hooks}
\newtcbtheorem[number within=section]{Exercise}{Exercise}
{%
	enhanced,
	breakable,
	colback = myexercisebg,
	frame hidden,
	boxrule = 0sp,
	borderline west = {2pt}{0pt}{myexercisefg},
	sharp corners,
	detach title,
	before upper = \tcbtitle\par\smallskip,
	coltitle = myexercisefg,
	fonttitle = \bfseries\sffamily,
	description font = \mdseries,
	separator sign none,
	segmentation style={solid, myexercisefg},
}
{th}

\tcbuselibrary{theorems,skins,hooks}
\newtcbtheorem[number within=chapter]{exercise}{Exercise}
{%
	enhanced,
	breakable,
	colback = myexercisebg,
	frame hidden,
	boxrule = 0sp,
	borderline west = {2pt}{0pt}{myexercisefg},
	sharp corners,
	detach title,
	before upper = \tcbtitle\par\smallskip,
	coltitle = myexercisefg,
	fonttitle = \bfseries\sffamily,
	description font = \mdseries,
	separator sign none,
	segmentation style={solid, myexercisefg},
}
{th}

%================================
% EXAMPLE BOX
%================================

\newtcbtheorem[number within=section]{Example}{Example}
{%
	colback = myexamplebg
	,breakable
	,colframe = myexamplefr
	,coltitle = myexampleti
	,boxrule = 1pt
	,sharp corners
	,detach title
	,before upper=\tcbtitle\par\smallskip
	,fonttitle = \bfseries
	,description font = \mdseries
	,separator sign none
	,description delimiters parenthesis
}
{ex}

\newtcbtheorem[number within=chapter]{example}{Example}
{%
	colback = myexamplebg
	,breakable
	,colframe = myexamplefr
	,coltitle = myexampleti
	,boxrule = 1pt
	,sharp corners
	,detach title
	,before upper=\tcbtitle\par\smallskip
	,fonttitle = \bfseries
	,description font = \mdseries
	,separator sign none
	,description delimiters parenthesis
}
{ex}

%================================
% DEFINITION BOX
%================================

\newtcbtheorem[number within=section]{Definition}{Definition}{enhanced,
	before skip=2mm,after skip=2mm, colback=red!5,colframe=red!80!black,boxrule=0.5mm,
	attach boxed title to top left={xshift=1cm,yshift*=1mm-\tcboxedtitleheight}, varwidth boxed title*=-3cm,
	boxed title style={frame code={
					\path[fill=tcbcolback]
					([yshift=-1mm,xshift=-1mm]frame.north west)
					arc[start angle=0,end angle=180,radius=1mm]
					([yshift=-1mm,xshift=1mm]frame.north east)
					arc[start angle=180,end angle=0,radius=1mm];
					\path[left color=tcbcolback!60!black,right color=tcbcolback!60!black,
						middle color=tcbcolback!80!black]
					([xshift=-2mm]frame.north west) -- ([xshift=2mm]frame.north east)
					[rounded corners=1mm]-- ([xshift=1mm,yshift=-1mm]frame.north east)
					-- (frame.south east) -- (frame.south west)
					-- ([xshift=-1mm,yshift=-1mm]frame.north west)
					[sharp corners]-- cycle;
				},interior engine=empty,
		},
	fonttitle=\bfseries,
	title={#2},#1}{def}
\newtcbtheorem[number within=chapter]{definition}{Definition}{enhanced,
	before skip=2mm,after skip=2mm, colback=red!5,colframe=red!80!black,boxrule=0.5mm,
	attach boxed title to top left={xshift=1cm,yshift*=1mm-\tcboxedtitleheight}, varwidth boxed title*=-3cm,
	boxed title style={frame code={
					\path[fill=tcbcolback]
					([yshift=-1mm,xshift=-1mm]frame.north west)
					arc[start angle=0,end angle=180,radius=1mm]
					([yshift=-1mm,xshift=1mm]frame.north east)
					arc[start angle=180,end angle=0,radius=1mm];
					\path[left color=tcbcolback!60!black,right color=tcbcolback!60!black,
						middle color=tcbcolback!80!black]
					([xshift=-2mm]frame.north west) -- ([xshift=2mm]frame.north east)
					[rounded corners=1mm]-- ([xshift=1mm,yshift=-1mm]frame.north east)
					-- (frame.south east) -- (frame.south west)
					-- ([xshift=-1mm,yshift=-1mm]frame.north west)
					[sharp corners]-- cycle;
				},interior engine=empty,
		},
	fonttitle=\bfseries,
	title={#2},#1}{def}



%================================
% Solution BOX
%================================

\makeatletter
\newtcbtheorem{question}{Question}{enhanced,
	breakable,
	colback=white,
	colframe=myb!80!black,
	attach boxed title to top left={yshift*=-\tcboxedtitleheight},
	fonttitle=\bfseries,
	title={#2},
	boxed title size=title,
	boxed title style={%
			sharp corners,
			rounded corners=northwest,
			colback=tcbcolframe,
			boxrule=0pt,
		},
	underlay boxed title={%
			\path[fill=tcbcolframe] (title.south west)--(title.south east)
			to[out=0, in=180] ([xshift=5mm]title.east)--
			(title.center-|frame.east)
			[rounded corners=\kvtcb@arc] |-
			(frame.north) -| cycle;
		},
	#1
}{def}
\makeatother

%================================
% SOLUTION BOX
%================================

\makeatletter
\newtcolorbox{solution}{enhanced,
	breakable,
	colback=white,
	colframe=myg!80!black,
	attach boxed title to top left={yshift*=-\tcboxedtitleheight},
	title=Solution,
	boxed title size=title,
	boxed title style={%
			sharp corners,
			rounded corners=northwest,
			colback=tcbcolframe,
			boxrule=0pt,
		},
	underlay boxed title={%
			\path[fill=tcbcolframe] (title.south west)--(title.south east)
			to[out=0, in=180] ([xshift=5mm]title.east)--
			(title.center-|frame.east)
			[rounded corners=\kvtcb@arc] |-
			(frame.north) -| cycle;
		},
}
\makeatother

%================================
% Question BOX
%================================

\makeatletter
\newtcbtheorem{qstion}{Question}{enhanced,
	breakable,
	colback=white,
	colframe=mygr,
	attach boxed title to top left={yshift*=-\tcboxedtitleheight},
	fonttitle=\bfseries,
	title={#2},
	boxed title size=title,
	boxed title style={%
			sharp corners,
			rounded corners=northwest,
			colback=tcbcolframe,
			boxrule=0pt,
		},
	underlay boxed title={%
			\path[fill=tcbcolframe] (title.south west)--(title.south east)
			to[out=0, in=180] ([xshift=5mm]title.east)--
			(title.center-|frame.east)
			[rounded corners=\kvtcb@arc] |-
			(frame.north) -| cycle;
		},
	#1
}{def}
\makeatother

\newtcbtheorem[number within=chapter]{wconc}{Wrong Concept}{
	breakable,
	enhanced,
	colback=white,
	colframe=myr,
	arc=0pt,
	outer arc=0pt,
	fonttitle=\bfseries\sffamily\large,
	colbacktitle=myr,
	attach boxed title to top left={},
	boxed title style={
			enhanced,
			skin=enhancedfirst jigsaw,
			arc=3pt,
			bottom=0pt,
			interior style={fill=myr}
		},
	#1
}{def}



%================================
% NOTE BOX
%================================

\usetikzlibrary{arrows,calc,shadows.blur}
\tcbuselibrary{skins}
\newtcolorbox{note}[1][]{%
	enhanced jigsaw,
	colback=gray!20!white,%
	colframe=gray!80!black,
	size=small,
	boxrule=1pt,
	title=\textbf{Note:-},
	halign title=flush center,
	coltitle=black,
	breakable,
	drop shadow=black!50!white,
	attach boxed title to top left={xshift=1cm,yshift=-\tcboxedtitleheight/2,yshifttext=-\tcboxedtitleheight/2},
	minipage boxed title=1.5cm,
	boxed title style={%
			colback=white,
			size=fbox,
			boxrule=1pt,
			boxsep=2pt,
			underlay={%
					\coordinate (dotA) at ($(interior.west) + (-0.5pt,0)$);
					\coordinate (dotB) at ($(interior.east) + (0.5pt,0)$);
					\begin{scope}
						\clip (interior.north west) rectangle ([xshift=3ex]interior.east);
						\filldraw [white, blur shadow={shadow opacity=60, shadow yshift=-.75ex}, rounded corners=2pt] (interior.north west) rectangle (interior.south east);
					\end{scope}
					\begin{scope}[gray!80!black]
						\fill (dotA) circle (2pt);
						\fill (dotB) circle (2pt);
					\end{scope}
				},
		},
	#1,
}

%%%%%%%%%%%%%%%%%%%%%%%%%%%%%%
% SELF MADE COMMANDS
%%%%%%%%%%%%%%%%%%%%%%%%%%%%%%


\newcommand{\thm}[2]{\begin{Theorem}{#1}{}#2\end{Theorem}}
\newcommand{\cor}[2]{\begin{Corollary}{#1}{}#2\end{Corollary}}
\newcommand{\mlemma}[2]{\begin{lemma}{#1}{}#2\end{lemma}}
\newcommand{\mprop}[2]{\begin{Prop}{#1}{}#2\end{Prop}}
\newcommand{\clm}[3]{\begin{claim}{#1}{#2}#3\end{claim}}
\newcommand{\wc}[2]{\begin{wconc}{#1}{}\setlength{\parindent}{1cm}#2\end{wconc}}
\newcommand{\thmcon}[1]{\begin{Theoremcon}{#1}\end{Theoremcon}}
\newcommand{\ex}[2]{\begin{Example}{#1}{}#2\end{Example}}
\newcommand{\dfn}[2]{\begin{Definition}[colbacktitle=red!75!black]{#1}{}#2\end{Definition}}
\newcommand{\dfnc}[2]{\begin{definition}[colbacktitle=red!75!black]{#1}{}#2\end{definition}}
\newcommand{\qs}[2]{\begin{question}{#1}{}#2\end{question}}
\newcommand{\pf}[2]{\begin{myproof}[#1]#2\end{myproof}}
\newcommand{\nt}[1]{\begin{note}#1\end{note}}

\newcommand*\circled[1]{\tikz[baseline=(char.base)]{
		\node[shape=circle,draw,inner sep=1pt] (char) {#1};}}
\newcommand\getcurrentref[1]{%
	\ifnumequal{\value{#1}}{0}
	{??}
	{\the\value{#1}}%
}
\newcommand{\getCurrentSectionNumber}{\getcurrentref{section}}
\newenvironment{myproof}[1][\proofname]{%
	\proof[\bfseries #1: ]%
}{\endproof}

\newcommand{\mclm}[2]{\begin{myclaim}[#1]#2\end{myclaim}}
\newenvironment{myclaim}[1][\claimname]{\proof[\bfseries #1: ]}{}

\newcounter{mylabelcounter}

\makeatletter
\newcommand{\setword}[2]{%
	\phantomsection
	#1\def\@currentlabel{\unexpanded{#1}}\label{#2}%
}
\makeatother




\tikzset{
	symbol/.style={
			draw=none,
			every to/.append style={
					edge node={node [sloped, allow upside down, auto=false]{$#1$}}}
		}
}


% deliminators
\DeclarePairedDelimiter{\abs}{\lvert}{\rvert}
\DeclarePairedDelimiter{\norm}{\lVert}{\rVert}

\DeclarePairedDelimiter{\ceil}{\lceil}{\rceil}
\DeclarePairedDelimiter{\floor}{\lfloor}{\rfloor}
\DeclarePairedDelimiter{\round}{\lfloor}{\rceil}

\newsavebox\diffdbox
\newcommand{\slantedromand}{{\mathpalette\makesl{d}}}
\newcommand{\makesl}[2]{%
\begingroup
\sbox{\diffdbox}{$\mathsurround=0pt#1\mathrm{#2}$}%
\pdfsave
\pdfsetmatrix{1 0 0.2 1}%
\rlap{\usebox{\diffdbox}}%
\pdfrestore
\hskip\wd\diffdbox
\endgroup
}
\newcommand{\dd}[1][]{\ensuremath{\mathop{}\!\ifstrempty{#1}{%
\slantedromand\@ifnextchar^{\hspace{0.2ex}}{\hspace{0.1ex}}}%
{\slantedromand\hspace{0.2ex}^{#1}}}}
\ProvideDocumentCommand\dv{o m g}{%
  \ensuremath{%
    \IfValueTF{#3}{%
      \IfNoValueTF{#1}{%
        \frac{\dd #2}{\dd #3}%
      }{%
        \frac{\dd^{#1} #2}{\dd #3^{#1}}%
      }%
    }{%
      \IfNoValueTF{#1}{%
        \frac{\dd}{\dd #2}%
      }{%
        \frac{\dd^{#1}}{\dd #2^{#1}}%
      }%
    }%
  }%
}
\providecommand*{\pdv}[3][]{\frac{\partial^{#1}#2}{\partial#3^{#1}}}
%  - others
\DeclareMathOperator{\Lap}{\mathcal{L}}
\DeclareMathOperator{\Var}{Var} % varience
\DeclareMathOperator{\Cov}{Cov} % covarience
\DeclareMathOperator{\E}{E} % expected

% Since the amsthm package isn't loaded

% I prefer the slanted \leq
\let\oldleq\leq % save them in case they're every wanted
\let\oldgeq\geq
\renewcommand{\leq}{\leqslant}
\renewcommand{\geq}{\geqslant}

% % redefine matrix env to allow for alignment, use r as default
% \renewcommand*\env@matrix[1][r]{\hskip -\arraycolsep
%     \let\@ifnextchar\new@ifnextchar
%     \array{*\c@MaxMatrixCols #1}}


%\usepackage{framed}
%\usepackage{titletoc}
%\usepackage{etoolbox}
%\usepackage{lmodern}


%\patchcmd{\tableofcontents}{\contentsname}{\sffamily\contentsname}{}{}

%\renewenvironment{leftbar}
%{\def\FrameCommand{\hspace{6em}%
%		{\color{myyellow}\vrule width 2pt depth 6pt}\hspace{1em}}%
%	\MakeFramed{\parshape 1 0cm \dimexpr\textwidth-6em\relax\FrameRestore}\vskip2pt%
%}
%{\endMakeFramed}

%\titlecontents{chapter}
%[0em]{\vspace*{2\baselineskip}}
%{\parbox{4.5em}{%
%		\hfill\Huge\sffamily\bfseries\color{myred}\thecontentspage}%
%	\vspace*{-2.3\baselineskip}\leftbar\textsc{\small\chaptername~\thecontentslabel}\\\sffamily}
%{}{\endleftbar}
%\titlecontents{section}
%[8.4em]
%{\sffamily\contentslabel{3em}}{}{}
%{\hspace{0.5em}\nobreak\itshape\color{myred}\contentspage}
%\titlecontents{subsection}
%[8.4em]
%{\sffamily\contentslabel{3em}}{}{}  
%{\hspace{0.5em}\nobreak\itshape\color{myred}\contentspage}



%%%%%%%%%%%%%%%%%%%%%%%%%%%%%%%%%%%%%%%%%%%
% TABLE OF CONTENTS
%%%%%%%%%%%%%%%%%%%%%%%%%%%%%%%%%%%%%%%%%%%

\usepackage{tikz}
\definecolor{doc}{RGB}{0,60,110}
\usepackage{titletoc}
\contentsmargin{0cm}
\titlecontents{chapter}[3.7pc]
{\addvspace{30pt}%
	\begin{tikzpicture}[remember picture, overlay]%
		\draw[fill=doc!60,draw=doc!60] (-7,-.1) rectangle (-0.9,.5);%
		\pgftext[left,x=-3.5cm,y=0.2cm]{\color{white}\Large\sc\bfseries Chapter\ \thecontentslabel};%
	\end{tikzpicture}\color{doc!60}\large\sc\bfseries}%
{}
{}
{\;\titlerule\;\large\sc\bfseries Page \thecontentspage
	\begin{tikzpicture}[remember picture, overlay]
		\draw[fill=doc!60,draw=doc!60] (2pt,0) rectangle (4,0.1pt);
	\end{tikzpicture}}%
\titlecontents{section}[3.7pc]
{\addvspace{2pt}}
{\contentslabel[\thecontentslabel]{2pc}}
{}
{\hfill\small \thecontentspage}
[]
\titlecontents*{subsection}[3.7pc]
{\addvspace{-1pt}\small}
{}
{}
{\ --- \small\thecontentspage}
[ \textbullet\ ][]

\makeatletter
\renewcommand{\tableofcontents}{%
	\chapter*{%
	  \vspace*{-20\p@}%
	  \begin{tikzpicture}[remember picture, overlay]%
		  \pgftext[right,x=15cm,y=0.2cm]{\color{doc!60}\Huge\sc\bfseries \contentsname};%
		  \draw[fill=doc!60,draw=doc!60] (13,-.75) rectangle (20,1);%
		  \clip (13,-.75) rectangle (20,1);
		  \pgftext[right,x=15cm,y=0.2cm]{\color{white}\Huge\sc\bfseries \contentsname};%
	  \end{tikzpicture}}%
	\@starttoc{toc}}
\makeatother


\begin{document}

\pdfbookmark[section]{\contentsname}{toc}
\tableofcontents

\setcounter{chapter}{1}

\chapter{Sobolev spaces}
\section{Interpolation inequalities}

\ex{}
{
    \begin{equation} \label{eq:1}
        \|u\|_{L^{2}}^{2} \leq \|u\|_{L^{2}} \|u'\|_{L^{2}} \text{ for } u \in C^{\infty}(\mathbb{R}) 
    \end{equation}
}

\begin{proof}
    Idea: use that \((u^2)' = 2uu'\) and Newton-Leibniz
    \begin{align*}
        u^2(x) &= 2 \int_{-\infty}^{x} u u' \,\mathrm{d}y = -2 \int_{x}^{\infty} u u' \,\mathrm{d}y \\
        &= \int_{-\infty}^{x} u u' \,\mathrm{d}y - \int_{x}^{\infty} u u' \,\mathrm{d}y \\
        & \leq \int_{-\infty}^{x} \vert u \vert \vert u' \vert \,\mathrm{d}y + \int_{x}^{\infty} \vert u \vert \vert u' \vert \,\mathrm{d}y \\
        &= \int_{\mathbb{R}} \vert u \vert \vert u' \vert \,\mathrm{d}y \\
        \text{(Hölder's inequality)} \quad & \leq \|u\|_{L^{2}} \|u'\|_{L^{2}}
    \end{align*}
\end{proof}

\qs{}
{
    Check that \ref{eq:1} is sharp. Namely, that \ref{eq:1} becomes equality for \(u(x) = e^{-\vert x \vert}\) (\(u(x)\) is an extremal function for \ref{eq:1}). Also, \ref{eq:1} is shift and scaling invariant, i.e. \(u_{\alpha}(x+h) = e^{-\alpha|x+h|}, h \in \mathbb{R}, \alpha>0\) -extremals.
}

\ex{Interpolation inequality}
{
    \(\Omega\)-domain in \(\mathbb{R}^{n}, u \in L_{p_1}(\Omega) \cap L_{p_2}(\Omega), 1 \leq p_1, p_2, < \infty, p_1 < p_2, \theta \in [0, 1], \frac{1}{p} = \frac{\theta}{p_1} + \frac{1-\theta}{p_2}\). Then
    \begin{equation}\label{eq:2}
        \|u\|_{L^{p}} \leq \|u\|_{L^{p_1}}^{\theta} \|u\|_{L^{p_2}}^{1 - \theta}
    \end{equation} 
}

\begin{proof}
    \[\int_{\mathbb{R}} \vert u \vert^p \,\mathrm{d}x = \int_{\mathbb{R}} \vert u \vert^{\theta p} \vert u \vert^{(1-\theta) p} \,\mathrm{d}x\]
    We apply Hölder's inequality with exponents \(P = \frac{p_1}{\theta p} \) and \(Q = \frac{p_2}{(1-\theta)p}\) (Note \(\frac{1}{P} + \frac{1}{Q} = \frac{\theta p}{p_1} + \frac{(1-\theta)p}{p_2} = 1\)). Then
    \begin{align*}
        \int_{\mathbb{R}} \vert u \vert^{\theta p} \vert u \vert^{(1-\theta) p} \,\mathrm{d}x &\leq \left(\int_{\mathbb{R}} \vert u \vert^{p_1} \,\mathrm{d}x \right)^{\frac{1}{P}} \left(\int_{\mathbb{R}} \vert u \vert^{p_2} \,\mathrm{d}x \right)^{\frac{1}{Q}} \\
        &= \|u\|_{L^{p_1}}^{\theta} \|u\|_{L^{p_2}}^{1 - \theta}
    \end{align*}
\end{proof}

\section{Sobolev inequalities}
\ex{Sobolev inequality 1D}
{
    \(u \in C^{\infty}([0,1])\), want to prove the embedding \(W^{1, 1}([0,1]) \subset C([0,1])\), i.e.
    \begin{equation}\label{eq:3}
        \|u\|_{C([0,1])} \leq \|u\|_{L^{1}([0,1])} + \|u'\|_{L^{1}([0,1])}
    \end{equation}
}

\begin{proof}
    By the Newton-Leibniz formula, \(u(x) - u(y) = \int_{y}^{x} u'(s) \,\mathrm{d}s\). Also,
    \[|u(x)| \leq |u(y)| + \int_{0}^{1} |u'(s)| \,\mathrm{d}s \quad \forall x, y \in [0,1]\]
    By integration over \(y \in [0,1]\), 
    \[|u(x)| \leq \int_{0}^{1} |u(s)| \,\mathrm{d}s + \int_{0}^{1} |u'(s)| \,\mathrm{d}s = \|u\|_{W^{1, 1}([0, 1])}\]

    Taking supremum with respect to \(x \in [0,1]\), we obtain \(\|u\|_{C([0,1])} \leq \|u\|_{W^{1, 1}([0, 1])}\)
\end{proof}

\ex{Sobolev inequality 2D}
{
    \(u \in C^{\infty}([0,1]^2)\), i.e. \(\Omega = [0,1]^2\), then \(W^{1, 1}(\Omega) \subset L^{2}(\Omega) : \|u\|_{L^{2}} \leq \|u\|_{W^{1, 1}(\Omega)}\)
}

\begin{proof}
    \(\int_{\Omega} u^{2}(x_1, x_2) \,\mathrm{d}x_1 \,\mathrm{d}x_2\) should be estimated. From \ref{eq:3}, we know that 
    \[\vert u(x_1, x_2) \vert \leq \int_{0}^{1} \vert u(s, x_2) \vert + |\partial_{x_1} u(s, x_2)| \,\mathrm{d}s \coloneqq f(x_2)\]
    \[\vert u(x_1, x_2) \vert \leq \int_{0}^{1} \vert u(x_1, s) \vert + |\partial_{x_2} u(x_1, s)| \,\mathrm{d}s \coloneqq g(x_1)\]
    Then 
    \begin{align*}
        \int_{\Omega} u^2 \,\mathrm{d}x &\leq \int_{0}^{1} g(x_1)f(x_2) \,\mathrm{d}x_1 \,\mathrm{d}x_2 \\
        &= \int_{0}^{1} f(x_2) \,\mathrm{d}x_2 \int_{0}^{1} g(x_1) \,\mathrm{d}x_1 \\
        &= \left(\int_{\Omega} \vert u(x_1, x_2) \vert + \vert \partial_{x_1} u(x_1, x_2) \vert \,\mathrm{d}x_1 \right) \left(\int_{\Omega} \vert u(x_1, x_2) \vert + \vert \partial_{x_2} u(x_1, x_2) \vert \,\mathrm{d}x_2 \right) \\
        &\leq \|u\|_{W^{1, 1}(\Omega)}
    \end{align*}
\end{proof}

\qs{Sobolev inequality 3D}
{
    \(u \in C^{\infty}(\bar{\Omega}), \Omega = (0,1)^3\). Prove that \(W^{1, 1}(\Omega) \subset L^{\frac{3}{2}}(\Omega)\), i.e.
    \begin{equation}\label{eq:4}
        \|u\|_{L^{\frac{3}{2}}(\Omega)} \leq \|u\|_{W^{1, 1}(\Omega)}
    \end{equation}
    Hint: first, prove that
    \[\int_{\Omega} f(x_1, x_2)g(x_2, x_3)h(x_1, x_3) \,\mathrm{d}x \leq \|f\|_{L^{2}} \|g\|_{L^{2}} \|h\|_{L^{2}}\]
    and use \ref{eq:3}.
}

\ex{}
{
    \(u \in C^{\infty}(\bar{\Omega}), \Omega = (0,1)^3\). Then 
    \begin{equation}
        \|u\|_{L^{6}(\Omega)} \leq C \|u\|_{W^{1, 2}(\Omega)}
    \end{equation}
}

\begin{proof}
    \begin{align*}
        \int_{\Omega} \vert u \vert^{6} \,\mathrm{d}x &= \int_{\Omega} (\vert u \vert^{4})^{\frac{3}{2}} \,\mathrm{d}x \\
        &\leq C \left(\int_{\Omega} \vert u \vert^{4} \,\mathrm{d}x + \int_{\Omega} u^{3}|\nabla u| \,\mathrm{d}x \right)^{\frac{3}{2}} \\
        (\text{by \eqref{eq:3}}) \quad &\leq C \left(\int_{\Omega} \vert u \vert^{4} \,\mathrm{d}x\right)^{\frac{3}{2}} + C \left(u^{3}|\nabla u| \,\mathrm{d}x \right)^{\frac{3}{2}} \\
        &\leq C \|u\|^{\frac{3}{2}\cdot \theta \cdot 4}_{L^{2}} \|u\|^{\frac{3}{2}\cdot (1-\theta) \cdot 4}_{L^{6}} + C \|u\|^{\frac{3}{2}\cdot 3}_{L^{6}} \|\nabla u\|^{\frac{3}{2}}_{L^{2}} \\
        \left(\theta = \frac{1}{4}\right) \quad &= C \|u\|^{\frac{3}{2}}_{L^{2}} \|u\|^{\frac{9}{2}}_{L^{6}} + C\|u\|^{\frac{9}{2}}_{L^{6}} \|\nabla u\|^{\frac{3}{2}}_{L^{2}} \\
        \left(\text{Young's inequality with } p=\frac{4}{5} \text{ and }q=-4\right) \quad &\leq \varepsilon \|u\|^{6}_{L^{6}} + C_{\varepsilon}(\|u\|_{L^{2}} + \|\nabla u\|_{L^{2}})^{6}
    \end{align*}
    Setting for example, \(\varepsilon = \frac{1}{2}\), we obtain
    \[\|u\|_{L^{6}(\Omega)} \leq C \|u\|_{W^{1, 2}(\Omega)}\]
\end{proof}

\begin{theorem}{Sobolev embeddings}{sobolev_embedding}
    \begin{enumerate}[label=\bfseries\tiny\protect\circled{\small\arabic*}]
		\item \(W^{k_1, p_1}(\Omega) \subset W^{k_2, p_2}(\Omega) \Longleftrightarrow k_1 \geq k_2\) and \(1 \leq p_1, p_2 < \infty, k_1 - \frac{n}{p_1} \geq k_2 - \frac{n}{p_2}, \Omega \subset \mathbb{R}^{n}\).
		\item \(W^{k,p}(\Omega) \subset C^{\alpha}(\Omega)\) if \(\alpha < k - \frac{n}{p}\). If \(\alpha\) is not an integer, then the inequality is weak.
	\end{enumerate}
\end{theorem}

\ex{}
{
    \(H^{s}(\mathbb{R}^{n}) \subset C(\mathbb{R}^{n}) \iff s > \frac{n}{2} \) 
}

\begin{proof}
    \(u(x) = \int_{\mathbb{R}^{n}} e^{i\xi x} \hat{u}(\xi) \,\mathrm{d}\xi \)
    \begin{align*}
        \vert u(x) \vert &\leq \int_{\mathbb{R}^{n}} \vert \hat{u}(\xi) \vert \,\mathrm{d}\xi \\
        &= \int_{\mathbb{R}^{n}} \left(1+\vert \xi \vert ^{2} \right)^{-\frac{s}{2}} \left(1+\vert \xi \vert ^{2} \right)^{\frac{s}{2}} \vert \hat{u}(\xi) \vert \,\mathrm{d}\xi \\
        \text{(Hölder's inequality)} \quad &\leq \left(\int_{\mathbb{R}^{n}} \frac{1}{\left(1+\vert \xi \vert ^{2}  \right)^{s}} \,\mathrm{d}\xi \right)^{\frac{1}{2}} \left(\int_{\mathbb{R}^{n}} \left(1+\vert \xi \vert ^{2} \right)^{s} \vert \hat{u}(\xi) \vert^{2} \,\mathrm{d}\xi \right)^{\frac{1}{2}}
    \end{align*} 

    \(\int_{\mathbb{R}^{n}} \frac{1}{\left(1+\vert \xi \vert ^{2} \right)^{s}} \,\mathrm{d}\xi < \infty \iff s>\frac{n}{2}\).
    Taking the supremum with respect to \(x \in \mathbb{R}^{n}\), we get
    \[\|u\|_{C(\mathbb{R}^{n})} \leq C_{s}\|u\|_{H^{s}(\mathbb{R}^{n})}\]
\end{proof}

\thm{Interpolation inequalities}
{
    Let \(u \in W^{k_1, p_1}(\Omega) \cap W^{k_2, p_2}(\Omega), \theta \in [0,1], 1\leq p_1, p_2 \leq \infty\) with \(k = \theta k_1 + (1-\theta) k_2, \frac{1}{p} = \frac{\theta}{p_1} + \frac{1-\theta}{p_2} \). Then 
    \[
        \|u\|_{W^{k, p}} \leq C\|u\|_{W^{k_1, p_1}}^{\theta} \|u\|_{W^{k_2, p_2}}^{1-\theta}
    \]
}

\cor{Particular cases}
{
    \begin{enumerate}
        \item \(\|u\|_{H^{1}} \leq \|u\|_{L^{2}}^{\frac{1}{2}} \|u\|_{H^{2}}^{\frac{1}{2}}\)
        \item \(\|u\|_{L^{p}} \leq \|u\|_{L^{p}}^{\theta} \|u\|_{H^{2}}^{1-\theta}\) 
    \end{enumerate}
}

\section{Spaces with zero boundary traces}
\dfn{}
{
    \(W^{1, p}_{0}(\Omega) \coloneqq \left\{u \in W^{1, p}(\Omega), \left. u \right|_{\partial \Omega} = 0\right\} \)

    An equivalent definition is that the Sobolev spaces \(W^{1, p}_{0}(\Omega)\) for \(1 \leq p < \infty\) are defined as the closure of the set of compactly supported test functions \(C_{0}^{\infty}(\Omega)\) with respect to the \(W^{1, p}(\Omega)\)-norm.
}

\mlemma{}
{
    These two definitions are equivalent. \(u \in \text{``closure"} \colon u = \lim\limits_{n \to \infty} \varphi_{n}, \varphi_{n} \in C^{\infty}_{0}(\Omega) \implies \left. \varphi_{n} \right|_{\partial \Omega} = 0\). By continuity, \(\left. u \right|_{\partial \Omega} = 0\). The proof of the converse statement is more technical and is omitted.
}

\section{Poincaré's and Friedrich's inequalities}
\mprop{Friedrich's inequality}
{
    Let \(\Omega\) be a bounded domain and \(u \in W^{1, p}_{0}(\Omega)\). Then
    \begin{equation}\label{eq:5}
        \|u\|_{L^{p}} \leq C \|\nabla u\|_{L^{p}} 
    \end{equation}
}

\begin{proof}
    It is enough to prove \ref{eq:5} for \(\varphi \in C^{\infty}_{0}(\Omega)\). By the Newton-Leibniz formula,
    \[    
        u(x_1, x') - u(-L, x') = u(x_1, x') = \int_{-L}^{x_1} \partial_{x_1}u(s, x') \,\mathrm{d}s 
    \] 

    \begin{align*}
        \vert u(x_1, x') \vert ^{p} &\leq \left(\int_{-L}^{L} \vert \partial_{x_1}u(s, x') \vert \,\mathrm{d}s \right)^{p} \\
        \text{(Hölder's inequality)} \quad & \leq C_{L} \int_{-L}^{L} \vert \partial_{x_1}u(s, x') \vert^{p} \,\mathrm{d}s
    \end{align*}
    Integration with respect to $x'$ gives us
    \[
        \int_{\mathbb{R}^{n-1}} \vert u(x_1, x') \vert ^{p} \,\mathrm{d}x' \leq C_{L}\|\partial_{x_1}u\|_{L^{p}}^{p}
    \]
    Finally, integrating over \(x_1 \in [-L, L]\), we obtain
    \[
        \|u\|_{L^{p}}^{p} \leq 2LC_{L} \|\partial_{x_1}u\|_{L^{p}}^{p}
    \]
\end{proof}

\cor{Equivalent norm in \(W^{1, p}_{0}(\Omega)\)}
{
    Homogeneous norm:
    \[
        \|u\|_{W^{1, p}_{0}(\Omega)} \coloneqq \|\nabla u\|_{L^{p}}
    \]
}

\begin{note}
    \(\left. u \right|_{\partial \Omega} = 0\) is important! Otherwise, \ref{eq:5} will fail for \(u \equiv c\). Since \(\nabla u\) defines \(u\) up to a constant; \(\left. u \right|_{\partial \Omega} = 0\) removes this constant.
\end{note}

\mprop{Poincaré inequality}
{
    Let \(\Omega\) be a bounded domain with a smooth boundary and \(\left\langle u \right\rangle \coloneqq \frac{1}{\vert \Omega \vert } \int_{\Omega} u(x) \,\mathrm{d}x = 0\). Then 
    \[
        \|u\|_{L^{p}} \leq C \|\nabla u\|_{L^{p}}
    \]   
}

\section{Compactness}
\begin{definition}{Sequential compactness}{}
    A metric space \((X, d)\) is compact if any sequence \(\{x_{n}\}_{n=1}^{\infty} \subset X\) has a convergent sub-sequence, i.e. there exists \(\{x_{n_{k}}\}_{k=1}^{\infty} \colon \lim\limits_{k \to \infty} x_{n_{k}} = x_0 \in X\).
\end{definition}

\begin{definition}{Compact}{}
    A topological space \(X\) is compact if any covering of \(X\) by open sets has a finite sub-covering.
\end{definition}

\begin{note}
    In metric spaces, compactness is equivalent to sequential compactness.

    In general topological spaces, they are not related.
\end{note}

\thm{Hausdorff}
{
    Let \((X, d)\) be a metric space. Then \(X\) is compact \(\iff\) \(X\) is complete and totally bounded.
}

\begin{definition}{Totally bounded}{}
    \(X\) is totally bounded if \(\forall \epsilon > 0, \exists\) covering of \(X\) by finitely many \(\epsilon\)-balls, i.e. \(X = \bigcup_{k=1}^{N} B_{\epsilon}(x_k), N = N(\epsilon)\) and \(\{x_k\}\) is an \(\epsilon\)-net in \(X\).
\end{definition}

\subsection*{Why do we need compactness?}
Let \(X\) be compact and \(f \colon X \to Y \) be continuous, then \(f(X)\) is compact in Y. How do we solve PDEs of the form (or more general equations)?
\begin{equation}\label{eq:6}
    F(x) = 0
\end{equation}
\begin{enumerate}
    \item Construct approximate solutions
    
    \(F(x_n) = g_n\), where \(\lim\limits_{n \to \infty} g_n = 0\)
    \item Obtain a priori estimates, i.e. that \(\{x_n\}\) is bounded in a proper space
    \item If \(\{x_n\}\) is pre-compact and \(F\) is continuous \(\implies x = \lim\limits_{x \to \infty} x_{n_{k}}\) is a solution of \ref{eq:6}.
\end{enumerate}

\thm{Arzelà-Ascoli}
{
    Let \(\Omega \subset \mathbb{R}^{n}\) be a bounded domain. Then \(V \subset C(\bar{\Omega})\) is compact iff:
    \begin{enumerate}
        \item \(V\)  is closed
        \item \(V\) is bounded
        \item \(V\) is equicontinuous \(= V\) has a common modulus of continuity
    \end{enumerate} 
}

\thm{Arzelà-Ascoli for \(L^p\)}
{
    Let \(\Omega \subset \mathbb{R}^{n}\) be a bounded domain, (and \(\partial \Omega\) smooth, although not needed), \(K \subset L^{p}(\Omega), 1 \leq p < \infty\). Then \(K\) is compact iff:
    \begin{enumerate}
        \item \(K\) is closed
        \item \(K\) is bounded
        \item \(K\) is equicontinuous in mean (possesses a joint modulus of continuity in \(L^p\)).
    \end{enumerate} 
}

\begin{definition}{Modulus of continuity}{}
    Let \(f \in L^p(\Omega), 1 \leq p < \infty, \Omega \subset \mathbb{R}^{n}\) bounded (\(\partial \Omega\) smooth not needed). \(\omega \colon \mathbb{R}^{+} \to \mathbb{R}^{+}\) such that \(\lim\limits_{z \to 0} w(z) = 0\) is a modulus of continuity of \(f\) in \(L_{p}(\Omega)\) if 
    \[
        \int_{\Omega} \vert f(x+h) - f(x) \vert ^p \,\mathrm{d}x \leq \omega(\vert h \vert ), \quad \forall h \in \mathbb{R}^n,
    \]
    where we used the \(0\)-extension of \(f\) outside of \(\Omega\). 
\end{definition}

\begin{corollary}{}{1}
    Let \(K = B_{1}(0) \in W^{1, p}(\Omega); \Omega \subset \mathbb{R}^{n}\) is bounded, \(\partial \Omega\) is smooth, \(1 \leq p < \infty\). Then \(K\) is pre-compact in \(L^{p}(\Omega)\).
\end{corollary}

\begin{proof}
    We need to check equicontinuity, i.e. estimate \(\int_{\Omega} \vert f(x+h) - f(x) \vert ^p \,\mathrm{d}x \).
    \[
        f(x+h) - f(x) = h \int_{0}^{1} \nabla f(x + sh) \,\mathrm{d}s
    \]
    Taking modulus and \(p\)-th power of both sides, we get
    \[
        \vert f(x+h) - f(x) \vert ^p \leq \vert h \vert \int_{0}^{1} \vert \nabla f(x + sh) \vert ^p \,\mathrm{d}s 
    \]
    Finally, we take an integral over \(x \in \Omega\).
    \begin{align*}
        \int_{\Omega} \vert f(x+h) - f(x) \vert ^p \,\mathrm{d}x &\leq \vert h \vert \int_{0}^{1} \int_{\Omega} \vert \nabla f(x + sh) \vert ^p \,\mathrm{d}x \,\mathrm{d}s \\
        &\leq C \vert h \vert
    \end{align*}
    \(\omega(z) = cz\) is a joint modulus of continuity.
\end{proof}

\begin{definition}{Compact embedding}{}
        Let \(V \subset W\) be Banach spaces. Then the embedding is compact if the unit ball of \(V\) is pre-compact in \(W\).
\end{definition}

\begin{note}
    We proved that \(W^{1, p}(\Omega) \subset L^{p}(\Omega)\) is a compact embedding.
\end{note}

\cor{}
{
    \(W^{1, p}(\Omega) \subset L^{q}(\Omega)\) is a compact embedding if \(q < q^{*}\), where \(q^{*}\) is defined such that \(\frac{1}{q^{*}} = \frac{1}{p} - \frac{1}{n}\) and \(\Omega \subset  \mathbb{R}^{n}\), \(\Omega\) is bounded, \(\partial \Omega\) is smooth.
}

\begin{proof}
    Let us check equicontinuity. 
    \[
        \|f(\cdot + h) - f(\cdot)\|_{L^{q}} \leq \|f(\cdot + h) - f(\cdot)\|_{L^{p}}^{\theta} \|f(\cdot + h) - f(\cdot)\|_{L^{q^{*}}}^{1-\theta} 
    \]
    since \(p < q < q^{*}\) and \(0 < \theta < 1\). \(q^{*}\) is a critical exponent in Sobolev embeddings, indeed, \(W^{1, p}(\Omega) \subset L^{q}(\Omega) \implies 1-\frac{n}{p} \geq -\frac{1}{q}\). Then by corollary \ref{cor:1}, we have
    \begin{align*}
        \|f(\cdot + h) - f(\cdot)\|_{L^{p}}^{\theta} \|f(\cdot + h) - f(\cdot)\|_{L^{q^{*}}}^{1-\theta} &\leq C \vert h \vert ^{\theta}(2\|f\|_{L^{q^{*}}})^{1-\theta} \\
        &\leq C_1 \vert h \vert ^{\theta} \|f\|_{W^{1, p}}^{1-\theta} \\
        &\leq C_1 \vert h \vert ^{\theta}
    \end{align*}
\end{proof}

General fact: \(W^{s_1, p_1}(\Omega) \subset W^{s_2, p_2}(\Omega)\), where \(\Omega\) is bounded, \(\partial \Omega\) is smooth. Embedding is compact \(\iff\) embedding is not critical.

\section{Dual spaces}
\begin{definition}{Dual space}{1}
    \(W^{-s, p}(\Omega) \coloneqq \left(W^{s, q}_{0}(\Omega)\right)^{*}\) is defined as the dual space to \(W^{s, q}_{0}(\Omega) =\), i.e. the space of linear continuous functionals on \(W^{s, q}_{0}(\Omega)\), where \(\frac{1}{p} + \frac{1}{q} = 1\).
\end{definition}

\begin{definition}{}{2}
    \(W^{-s, p}(\Omega) = \left\{\text{completion of } L^{p}(\Omega) \text{ w.r.t } \|\ell\|_{W^{-s, p}} \coloneqq \sup\limits_{\varphi \in \mathcal{D}} \frac{\vert (\ell, \varphi) \vert }{\|\varphi\|_{W^{s, q}_{0}}} \right\} \) 
\end{definition}

\begin{definition}{}{3}
    \(W^{-s, p}(\Omega) = \left\{\ell \in \mathcal{D}'(\Omega) : \|\ell\|_{W^{-s, p}} \coloneqq \sup\limits_{\varphi \in \mathcal{D}} \frac{\vert\left\langle \ell, \varphi \right\rangle\vert}{\|\varphi\|_{W^{s, q}_{0}}} \right\} \)
\end{definition}

\mprop{}
{
    Definitions \ref{def:1}, \ref{def:2} and \ref{def:3} are equivalent.
}

\begin{question}{}{}
    Suppose \(\delta(x) \in W^{-s, p}(\Omega), \Omega \subset \mathbb{R}^{n}\). How are \(s, p\) and \(n\) related? We know that \(\delta(x)\) is well-defined on continuous functions, so we need \(W^{s, q}_{0}(\Omega) \subset C(\bar{\Omega})\).
\end{question}

\begin{example}{}{}
    Consider the case where \(n=1\) and \(p=2\). By the Sobolev embedding theorem, \(W^{s, 2} \subset C(\bar{\Omega})\) if \(0 < s - \frac{1}{2}\). 
    Thus we have \(\delta(x) \in H^{-s}(\Omega)\) if \(s>\frac{1}{2}\).
\end{example}

\chapter{Linear elliptic problems}
\section{Dirichlet and Neumann problems for the Laplacian}

\begin{example}{Laplace equation with Dirichlet boundary conditions}{}
    Let \(\Omega \in \mathbb{R}^{n}\) be a bounded domain with \(\partial \Omega\) smooth. Consider the Laplace equation with Dirichlet boundary conditions:
    \begin{equation}\label{eq:3.1}
        \begin{cases}
            \Delta u = f \\
            \left. u \right|_{\partial \Omega} = 0
        \end{cases}
    \end{equation}
    Typical questions: 
    \begin{enumerate}
        \item In what space does the solution live?
        \item In what sense is the equation understood (classical / weak)?
        \item In what sense are the boundary / initial data understood?
    \end{enumerate}
    In ODEs, we have local existence and uniqueness theorem (for Lipschitz non-linearities), but there is not an equivalent theorem for PDEs. Therefore, we must study particular examples.
\end{example}

\begin{definition}{}{}
    \(u \in W^{1, 2}_{0}(\Omega)\) is a weak solution of \ref{eq:3.1} if \(\forall \varphi \in C^{\infty}_{0}(\Omega)\),
    \begin{equation}\label{eq:3.2}
        - \int_{\Omega} \nabla u(x) \nabla \varphi(x) \,\mathrm{d}x = \ \int_{\Omega} f(x) \varphi(x) \,\mathrm{d}x
    \end{equation}
    Here, the boundary condition is incorporated into the choice of space \(W^{1, 2}_{0}(\Omega) = [C^{\infty}_{0}(\Omega)]_{W^{1, 2}(\Omega)}\) (the closure of \(C^{\infty}_{0}(\Omega)\) in the norm of \(W^{1, 2}(\Omega)\)). 

    \ref{eq:3.2} came from the integration by parts formula. Indeed, if \(u \in C^{2}(\Omega) \cap C(\bar{\Omega})\), then \(\Delta u = f\) is understood in a classical sense and
    \[
        \int_{\Omega} \Delta u \varphi \,\mathrm{d}x = - \int_{\Omega} \nabla u \nabla \varphi \,\mathrm{d}x + \int_{\partial\Omega} \partial_{n} u \varphi \,\mathrm{d}s,
    \]
    where the term \(\int_{\partial\Omega} \partial_{n} u \varphi \,\mathrm{d}s = 0\) because \(\left. \varphi \right|_{\partial \Omega} = 0\).
\end{definition}

\begin{theorem}{}{}
    Let \(f \in H^{-1}(\Omega) \coloneqq W^{-1, 2}(\Omega)\). Then \ref{eq:3.1} has a unique weak solution.
\end{theorem}

\begin{proof}{Application of Riesz representation theorem}

    \([u,u] \coloneqq \int_{\Omega} \nabla u \nabla u \,\mathrm{d}x\) is an equivalent norm on \(W^{1, 2}_{0}(\Omega)\) (due to Friedrich's inequality). Then \ref{eq:3.2} can be rewritten as
    \[
        [u, \varphi] = \int_{\Omega} f(x)\varphi(x) \,\mathrm{d}x \coloneqq \ell(\varphi)
    \]

    Claim: \(\ell\) is a linear continuous functional on \(W^{1, 2}_{0}(\Omega)\) (the integral should be understood as duality if we take \(f \in H^{-1}(\Omega)\) and if \(f \in L^{2}(\Omega)\), this is a standard Lebesgue integral).

    Linearity of \(\ell\) is obvious. \(\ell\) is continuous as it is bounded:
    \[
        \vert \ell(\varphi) \vert \leq \|f\|_{H^{-1}} \| \varphi \|_{H^{1}}
    \]
    But we obtained that \ref{eq:3.2} holds only for \(\varphi \in C^{\infty}_{0}(\Omega)\), not for \(\varphi \in W^{1, 2}_{0}(\Omega)\). However, \(W^{1, 2}_{0}(\Omega) = [C^{\infty}_{0}(\Omega)]_{W^{1, 2}}\). Then approximation arguments give that \(\forall \varphi \in H\), 
    \begin{equation}\label{eq:3.3}
        [u, \varphi] = \ell(\varphi)
    \end{equation}
    Then by Riesz representation theorem, there exists a unique \(u \in W^{1, 2}_{0}(\Omega)\) which satisfies \ref{eq:3.3}.
\end{proof}

\begin{example}{Laplace equation with Neumann boundary conditions}{}
    Let \(\Omega \in \mathbb{R}^{n}\) be a bounded domain with \(\partial \Omega\) smooth. Consider the Laplace equation with Neumann boundary conditions:
    \begin{equation}\label{eq:3.4}
        \begin{cases}
            \Delta u = f \\
            \left. \partial_{n} u \right|_{\partial \Omega} = 0
        \end{cases}
    \end{equation}
    We cannot consider \(\varphi \in C^{\infty}_{0}(\Omega)\) as test functions, because the information about boundary conditions will be lost. Similarly, considering
    \[
        \varphi \in W^{1, 2}_{n}(\Omega) \coloneqq \{u \in W^{1, 2}(\Omega) \colon \left. \partial_{n} u \right|_{\partial \Omega} = 0\}
    \]
    will not work as well, since \(\left. \partial_{n} u \right|_{\partial \Omega}\) is not defined for \(u \in W^{1, 2}(\Omega)\) (since by theorem \ref{th:sobolev_embedding}, \(C^{\infty}(\Omega) \not\subset W^{1, 2}(\Omega)\)). Instead, let us take \(\varphi \in C^{\infty}(\bar{\Omega})\) as a test function and assume that \(u\) is a classical solution. Then
    \begin{align*}
        \int_{\Omega} f \varphi \,\mathrm{d}x &= \int_{\Omega} \Delta  u \varphi \,\mathrm{d}x \\
        &= - \int_{\Omega} \nabla u \nabla \varphi \,\mathrm{d}x + \int_{\partial \Omega} \partial_{n} u \varphi \,\mathrm{d}s \\
        &= - \int_{\Omega} \nabla u \nabla \varphi \,\mathrm{d}x ,
    \end{align*}
    as \(\int_{\partial \Omega} \partial_{n} u \varphi \,\mathrm{d}x = 0\) due to the boundary conditions. If we take \(\varphi(x) = 1\) as a test function, then we get
    \begin{align*}
        \int_{\Omega} f \cdot 1 \,\mathrm{d}x &= - \int_{\Omega} \nabla u \nabla 1 \,\mathrm{d}x \\
        &= 0
    \end{align*}
    Hence \(\left\langle f \right\rangle = \frac{1}{\vert \Omega \vert }\int_{\Omega} f(x) \,\mathrm{d}x = 0\) is a necessary condition for solvability.

    Let us notice that all solutions of this problem differs from each other by a constant. Thus, a natural assumption to single out the solution is \(\left\langle u \right\rangle = 0\). 
\end{example}

\begin{definition}{}{}
    \(u \in W^{1, 2}(\Omega) \cap \{\left\langle u \right\rangle = 0\}\) is a weak solution of \ref{eq:3.4} if \(\forall \varphi \in C^{\infty}(\bar{\Omega})\), we have:
    \begin{equation}\label{eq:3.5}
        \int_{\Omega} \nabla u \nabla \varphi \,\mathrm{d}x = - \int_{\Omega} f \varphi \,\mathrm{d}x
    \end{equation}

\end{definition}

\begin{note}
    The boundary conditions are now not in the definition of the space, but in \ref{eq:3.5}.
\end{note}

\begin{theorem}{}{}
    Let \(f \in L^{2}(\Omega) \cap \{\left\langle f \right\rangle = 0\}\). Then \ref{eq:3.4} has a unique weak solution. 
\end{theorem}

\begin{proof}
    The proof is analogous to the problem with Dirichlet boundary conditions, but instead of applying Friedrich's inequality, we should apply Poincaré's inequality and use density of \(C^{\infty}(\Omega) \in W^{1, 2}(\Omega)\). 
\end{proof}

\begin{example}{Non-homogeneous Neumann boundary conditions}{}
    Let \(\Omega \in \mathbb{R}^{n}\) be a bounded domain with \(\partial \Omega\) smooth. Consider the Laplace equation with non-homogeneous Neumann boundary conditions:
    \begin{equation}\label{eq:3.6}
        \begin{cases}
            \Delta u = f \\
            \left. \partial_{n} u \right|_{\partial \Omega} = g
        \end{cases}
    \end{equation}
\end{example}

\begin{definition}{}{}
    \(u \in W^{1, 2}(\Omega) \cap \{\left\langle u \right\rangle = 0\}\) is a weak solution of \ref{eq:3.6} if \(\forall \varphi \in C^{\infty}(\bar{\Omega})\), we have:
    \begin{equation}\label{eq:3.7}
        \int_{\Omega} \nabla u \nabla \varphi \,\mathrm{d}x = - \int_{\Omega} f \varphi \,\mathrm{d}x + \int_{\partial\Omega} g\varphi \,\mathrm{d}s
    \end{equation}
    Note that if \(\varphi \equiv 1\), then a necessary condition for solvability is
    \[
        - \int_{\Omega} f \,\mathrm{d}x + \int_{\partial\Omega} g \,\mathrm{d}s = 0
    \]
\end{definition}

\begin{theorem}{}{}
    Let \(f \in L^{2}(\Omega), g \in W^{-\frac{1}{2}, 2}(\partial \Omega)\) be such that \(\int_{\Omega} f \,\mathrm{d}x = \int_{\partial\Omega} g \,\mathrm{d}s\). Then \ref{eq:3.6} has a unique weak solution.  
\end{theorem}

\begin{proof}
    \([u, u] \coloneqq \int_{\Omega} \nabla u \nabla u ds\) is an equivalent norm on \(u \in W^{1, 2}(\Omega) \cap \{\left\langle u \right\rangle = 0\}\) due to the Poincaré inequality. Then \ref{eq:3.7} can be rewritten as
    \[
        [u, \varphi] = \ell(\varphi) \coloneqq - \int_{\Omega} f\varphi\,\mathrm{d}x + \int_{\partial\Omega} g\varphi\,\mathrm{d}s
    \]
    We claim that \(\ell\) is a linear continuous functional on \(W^{1, 2}(\Omega) \cap \{\left\langle u \right\rangle = 0\}\). Indeed, linearity is obvious. To show \(\ell\) is  continuous, we have
    \begin{align*}
        \left\vert - \int_{\Omega} f\varphi\,\mathrm{d}x + \int_{\partial\Omega} g\varphi\,\mathrm{d}s \right\vert &\leq \|f\|_{L^{2}} \|\varphi\|_{L^{2}} + \|g\|_{H^{-\frac{1}{2}}(\partial \Omega)}\|\varphi\|_{H^{\frac{1}{2}}(\partial \Omega)} \\
        \text{(By the trace theorem and Poincaré's inequality)} &\leq \|f\|_{L^{2}} \|\varphi\|_{W^{1, 2}(\Omega)} + \|g\|_{H^{-\frac{1}{2}}(\partial \Omega)}\|\varphi\|_{W^{1, 2}(\Omega)}  
    \end{align*}

Then by Riesz representation theorem, there exists a unique \(u \in W^{1, 2}(\Omega) \cap \{\left\langle u \right\rangle = 0\}\) that is a weak solution of \ref{eq:3.6}.
\end{proof}

\begin{example}{Non-homogeneous Dirichlet boundary conditions}{}
    Let \(\Omega \in \mathbb{R}^{n}\) be a bounded domain with \(\partial \Omega\) smooth. Consider the Laplace equation with non-homogeneous Dirichlet boundary conditions:
    \begin{equation}\label{eq:3.8}
        \begin{cases}
            \Delta u = 0 \\
            \left. u \right|_{\partial \Omega} = g
        \end{cases}
    \end{equation}
    Let us take \(g \in W^{\frac{1}{2}, 2}(\partial \Omega)\). Then there exists \(v \in W^{1, 2}(\Omega)\) such that \(\left. v \right|_{\partial \Omega} = g\) (by the trace theorem). We look for the solution of \ref{eq:3.8} in the form \(u = v + w\), where \(w \in W^{1, 2}_{0}(\Omega)\).
\end{example}

\begin{definition}{}{}
    \(u = v + w\) is a weak solution of \ref{eq:3.8} if \(\left. v \right|_{\partial \Omega} = g\), where \(g \in W^{\frac{1}{2}, 2}(\partial \Omega), w \in W^{1, 2}_{0}(\Omega)\) and \(\forall \varphi \in C^{\infty}(\bar{\Omega})\), we have 
    \begin{equation}\label{eq:3.9}
        \int_{\Omega} \nabla (v+w) \nabla \varphi \,\mathrm{d}x = 0
    \end{equation}    
\end{definition}

\begin{theorem}{}{}
    Let \(g \in W^{\frac{1}{2}, 2}(\partial \Omega)\). Then \ref{eq:3.8} has a unique weak solution.
\end{theorem}

\begin{proof}
    We can rearrange \ref{eq:3.9} to get
    \[
        \ell(\varphi) \coloneqq -[v, \varphi] = \int_{\Omega} \nabla w \nabla \varphi \,\mathrm{d}x = [w, \varphi],
    \]
    and the functional \(\ell\) can be shown to be linear and continuous. By the Riesz representation theorem, there exists a unique \(w \in W^{1, 2}(\Omega)\) such that \ref{eq:3.9} is satisfied. Note that this \(w\) depends on the choice of \(v\). But \(u = v + w\) does not depend on the choice of \(v\). Indeed, let \(u_1\) and \(u_2\) be two solutions of \ref{eq:3.8}. Then \(u = u_1 - u_2\) solves
    \[
        \begin{cases}
            \Delta u = 0 \\
            \left. u \right|_{\partial \Omega} = 0
        \end{cases}
    \]
    We have previously shown that the weak solution of this problem is unique. Therefore, \(u_1 = u_2\). 
\end{proof}

\begin{note}
    There is no universal choice of the space of test functions. Even for Dirichlet and Neumann boundary conditions, we need to consider different spaces. \(\varphi \in C^{\infty}_{0}(\Omega)\) corresponds to the standard theory of distributions, while \(\varphi \in C^{\infty}(\bar{\Omega})\) corresponds to ``non-standard'' distributions.
\end{note}

\begin{example}{}{}
    Let \(\Omega \in \mathbb{R}^{n}\) be a bounded domain with \(\partial \Omega\) smooth. Consider
    \begin{equation}\label{eq:3.10}
        \begin{cases}
            \sum_{i, j} \partial_{x_{i}}(a_{ij}(x)\partial_{x_{j}}u) = g \\
            \left. u \right|_{\partial \Omega} = 0
        \end{cases}
    \end{equation}
    Where we make the following assumptions on the matrix \(a(x) \coloneqq \{a_{ij}(x)\}_{i,j}\):
    \begin{enumerate}
        \item \(a(x)\) is a symmetric matrix for every \(x\): 
        \[a_{ij}(x) = a_{ji}(x)\]
        \item \(a(x)\) is uniformly elliptic. That is, for all \(\xi \in \mathbb{R}^{n}\), there exists \(\mu, M > 0\) which are independent of \(x\) such that
        \[
            \mu \vert \xi^{2} \vert \leq \sum_{i,j} a_{ij} \xi_{i} \xi_{j} \leq M \vert \xi^{2} \vert 
        \]
    \end{enumerate} 
\end{example}

\begin{definition}{}{}
    \(u \in W^{1, 2}(\Omega)\) is a weak solution to \ref{eq:3.10} \(\iff \forall \varphi \in C^{\infty}_{0}(\Omega)\), we have 
    \[
        \sum_{i,j} \int_{\Omega} a_{ij} \partial_{x_{j}} u \partial_{x_{i}} \varphi \,\mathrm{d}x = -\int_{\Omega} g\varphi \,\mathrm{d}x
    \]
\end{definition}

\begin{theorem}{}{}
    Let \(a(x)\) be symmetric and uniformly elliptic. Then \ref{eq:3.10} has a unique weak solution.
\end{theorem}

\begin{proof}
    Let us denote
    \[
        [u, \varphi]_{a} = \int_{\Omega} \sum_{i,j} a_{ij}(x) \partial_{x_{j}} u(x) \partial_{x_{i}} \varphi(x) \,\mathrm{d}x.
    \]
    Then since \(a(x)\) is symmetric, the bilinear form \([u, v]_{a}\) is also symmetric, i.e. \([u, v]_{a} = [v, u]_{a}\). Since \(a(x)\) is uniformly elliptic, there exist \(\mu, M > 0\) such that
    \[
        \mu[u, u] \leq [u, u]_{a} \leq M[u, u].
    \]
    Therefore, \(\left(W^{1, 2}_{0}(\Omega), [\cdot, \cdot]_{a}\right)\) is a Hilbert space with the norm equivalent to the standard \(W^{1, 2}_{0}(\Omega)\) norm.

    By the Riesz representation theorem, there exists a unique weak solution to \ref{eq:3.10}.
\end{proof}

\section{More general problems via Lax-Milgram}
By the Riesz representation theorem, for any linear continuous functional, \(\ell\) on a Hilbert space \(H\), there exists a unique \(x \in H\) such that \(\forall \varphi \in  H\), we have \((x, \varphi) = \ell(\varphi)\).

If we want \(a(x, y)\) to be an equivalent inner product on \(H\), then \(a(x, y)\) must be symmetric.

We now consider the case where \(a(x, y)\) is not assumed to be symmetric.

\begin{definition}{Bilinear form}{}
    A bilinear form \(a(\cdot, \cdot) \colon H \times H \rightarrow \mathbb{R}\) is bounded if 
    \[
        \vert a(x, y) \vert \leq C \|x\|\|y\|
    \]
\end{definition}

\begin{definition}{Coercive}{}
    A bilinear form \(a(\cdot, \cdot)\) is coercive if \(\exists \alpha > 0\) such that \(a(x, x) \geq \alpha\|x\|^{2}\). 
\end{definition}

\begin{theorem}{}{}
    Let \(a(x, y)\) be a bounded and coercive bilinear form on \(H\). Then any linear continuous functional \(\ell \colon H \to \mathbb{R}\) can be represented in the form
    \begin{equation}\label{eq:3.11}
        a(x, y) = \ell(\varphi), \quad \forall \varphi \in H.
    \end{equation}
    i.e. \(\forall \ell \in H^{*}\), there exists a unique \(x = x(\ell) \in H\) such that \ref{eq:3.11} is satisfied.
\end{theorem}

\begin{example}{}{}
    Let \(\Omega \in \mathbb{R}^{n}\) be a bounded domain with \(\partial \Omega\) smooth. Consider the problem
    \begin{equation}\label{eq:3.12}
        \begin{cases}
            \sum_{i, j} \partial_{x_{i}}(a_{ij}(x)\partial_{x_{j}}u) + \sum_{i}b_{i}(x) \partial_{x_{i}}u = g(x) \\
            \left. u \right|_{\partial \Omega} = 0
        \end{cases}
    \end{equation}
\end{example}

\begin{definition}{}{}
    \(u \in W^{1, 2}_{0}(\Omega)\) is a weak solution of \ref{eq:3.12} if \(\forall \varphi \in C^{\infty}_{0}(\Omega)\), we have 
    \[
        A(u, \varphi) \coloneqq \sum_{i, j} \int_{\Omega} a_{ij} \partial_{x_{j}}u \partial_{x_{i}}\varphi \,\mathrm{d}x - \sum_{i} \int_{\Omega} b_{i}(x) \partial_{x_{i}} u \varphi \,\mathrm{d}x = \ell(\varphi) \coloneqq - \int_{\Omega} g(x)\varphi(x) \,\mathrm{d}x
    \]
\end{definition}

\begin{theorem}{}{}
    Let \(\{a_{ij}\} \in L^{\infty}(\Omega)\) be a uniformly elliptic matrix, \(b_{i}(x)\) be a smooth divergent free vector field and \(g(x) \in H^{-1}(\Omega)\). Then \ref{eq:3.12} has a unique weak solution.
\end{theorem}

\begin{proof}
    We use the Lax-Milgram theorem. We know that \(\ell(\varphi)\) is a linear continuous functional on \(W^{1, 2}_{0}(\Omega)\). Furthermore, \(A(u, \varphi)\) is bilinear and bounded. Indeed, by Friedrich's inequality, we have
    \[
        \vert A(u, \varphi) \vert \leq C_{1}\|\nabla u\|_{L^{2}} \|\nabla \varphi\|_{L^{2}} + C_2\|\nabla u\|_{L^{2}}\|\varphi\|_{L^{2}} \leq C\|u\|_{W^{1, 2}_{0}(\Omega)}\|\varphi\|_{W^{1, 2}_{0}(\Omega)}
    \]
    \(A(u, \varphi)\) is coercive since 
    \begin{align*}
        A(u, u) &= \sum_{i, j} \int_{\Omega} a_{ij} \partial_{x_{j}}u \partial_{x_{i}}u \,\mathrm{d}x - \sum_{i} \int_{\Omega} b_{i}(x) \partial_{x_{i}} (u) u \,\mathrm{d}x \\
        &\geq \alpha \|\nabla u\|_{L^{2}}^{2} - \frac{1}{2}\int_{\Omega} \sum_{i} b_{i}(x) \partial_{x_{i}} (u^{2}) \,\mathrm{d}x \\
        &= \alpha \|\nabla u\|_{L^{2}}^{2} + \frac{1}{2} \int_{\Omega} \text{div}b \cdot u^{2}(x) \,\mathrm{d}x \\
        &= \alpha\|\nabla u\|_{L^{2}}^{2}
    \end{align*}
    By the Lax-Milgram theorem, there exists a unique weak solution of \ref{eq:3.12}.
\end{proof}

\section{Introduction to spectral theory}
\(H\) is a Hilbert space. \(\mathcal{L}(H)\) is a space of linear continuous operators.

\begin{lemma}{}{}
    \(A\) is continuous \(\iff\) A is bounded, i.e.
    \[
        \|A\| \coloneqq  \sup_{x \in H} \frac{\|Ax\|}{\|x\|} < \infty
    \]
\end{lemma}

\begin{lemma}{}{}
    \((\mathcal{L}(H), \|\cdot\|)\) is a Banach space. 
\end{lemma}

\begin{definition}{Invertible operator}{}
    \(A\) is invertible \(\iff \exists A^{-1} \in \mathcal{L}(H)\) such that
    \begin{equation}\label{eq:3.13}
        AA^{-1} = A^{-1}A = I
    \end{equation}
\end{definition}

\begin{definition}{Spectrum}{}
    \(\lambda \in \sigma(A)\) (the spectrum of \(A\)) \(\iff\) \(\lambda I - A\) is not invertible.

    In other words, \(\lambda \notin \sigma(A)\), (\(\lambda\) is in the resolvent set) iff the equation \(\lambda u - Au = f\) has a unique solution for all \(f \in H\).
\end{definition}

\begin{note}
    If \(\dim H = n < \infty\), then \(\mathcal{L}(H) = M(n \times n)\) (\(n \times n\) matrices) and
    \begin{enumerate}
        \item All linear operators are continuous,
        \item All \(\lambda\in \sigma(A)\) correspond to eigenvalues
        \[
            A \rho_{\lambda} = \lambda \rho_{\lambda} \implies \sigma(A) = \sigma_{p}(A)\text{ (point spectrum).}
        \]
        \item \(\lambda \in \sigma(A) \iff \det(\lambda I - A) = 0\)
        \item Only one of two equalities from \ref{eq:3.13} holding is enough. 
    \end{enumerate}
    All of the statements may fail when \(\dim H = \infty\).
\end{note}

\begin{example}{}{}
    Let \(H = \ell_{2}\), the space of square summable sequences. Let \(T_{r}, T_{l} \in \mathcal{L}(H)\) denote the right and left shift operators, respectively;
    \begin{align*}
        T_{r}(x_1, x_2, x_3, \dots) &= (0, x_1, x_2, x_3, \dots) \\
        T_{l}(x_1, x_2, x_3, \dots) &= (x_2, x_3, \dots)
    \end{align*}
    Then \(T_{l} \circ T_{r} = I\), but \(T_{r} \circ T_{l} (x) = (0, x_2, x_3, \dots)\). Hence \(T_{r}\) has a left inverse, but not a right inverse. \(\ker(T_{r}) = \{0\}\), i.e. it is injective (no eigenvalues), but the range of \(T_{r} \neq H\), since \(T_{r}(x) \perp e_{1}\) for any \(x \in H\). So \(T_{r}\) is not invertible since \(T_{r}(H)\) is a proper closed subspace of \(H\). Therefore, a new type of spectrum appeared - the residual spectrum \(\sigma_{R}(A)\), which is impossible in finite-dimensional spaces. 

    Also, the determinant does not exist in infinite-dimensional spaces. Indeed, if it existed, then
    \[
        1 = \det(T_{l} \circ T_{r}) = \det(T_{l})\det(T_{r}),
    \]
    which implies that both \(T_{l}\) and \(T_{r}\) are invertible, but this is not true.
\end{example}

\begin{definition}{Approximate point spectrum}{}
    \(\lambda \in \sigma_{\text{app}}(A)\) (approximate point spectrum) \(\iff \exists x_{n} \in H, \|x_{n}\| = 1\) and \(\lim\limits_{n \rightarrow \infty} (Ax_{n}-\lambda x_{n}) = 0\).

    It can be proved that \(\lambda \in \sigma_{\text{app}}(A) \iff \) the image of \((\lambda I - H)\) is not closed.
\end{definition}

\begin{example}{}{}
    \(H = L^{2}(0, 1), Af(x) \coloneqq xf(x)\).
    \[
        \lambda u - Au = f \iff (\lambda - x)u(x) = f(x),
    \]
    then \(u(x) = \frac{f(x)}{\lambda - x}\) and \((A - \lambda I)\) is invertible \(\iff \lambda \in [0, 1] \implies \sigma(A) = [0, 1]\).

    We can check that Range(\(\lambda I - A\)) is not closed if \(\lambda \in [0, 1]\).
\end{example}
The next theorem shows that these are all the possible obstacles to invert the operator.

\begin{theorem}{Weyl's theorem}{}
    \(A \in \mathcal{L}(H)\). Then \(\sigma(A) = \sigma_{p}(A) \cup \sigma_{\text{app}}(A) \cup \sigma_{R}(A)\).
\end{theorem}

\begin{definition}{Resolvent}{}
    \(R_{A}(\lambda) := (\lambda I - A)^{-1}\) is called the resolvent of \(A \in \mathcal{L}(H)\).
\end{definition}

More standard facts:
\begin{enumerate}
    \item If \(\vert \lambda \vert > \|A\|\), then \(\lambda \in \sigma(A)\). Indeed, 
    \[
        \frac{1}{\lambda - A} = \frac{1}{\lambda} \frac{1}{1-\frac{A}{\lambda}} = \frac{1}{\lambda} \sum_{n=0}^{\infty} \left(\frac{A}{n}\right)^{n}
    \]
    is an absolutely convergent series if \(\vert \lambda \vert  > \|A\| \).
    \item Resolvent identity:
    \begin{equation}\label{eq:3.14}
        R_{A}(\lambda) - R_{A}(\mu) = -(\lambda - \mu)R_{A}(\lambda)R_{A}(\mu),
    \end{equation}
    which follows from 
    \[
        \frac{1}{\lambda - x} - \frac{1}{\lambda - \mu} = (\mu - \lambda) \frac{1}{\lambda - x} \frac{1}{\mu - x},
    \]
    where we substitute in \(x = A\).
    From \ref{eq:3.14}, by taking the limit as \(\mu \to \lambda\), we get
    \[
        \frac{d}{d \lambda} R_{A}(\lambda) = -R_{A}(\lambda)^{2}.
    \]
    Hence \(R_{A}(\lambda)\) is an analytic function of \(\lambda\).
    \item By Liouville's theorem applied to \(R_{A}(\lambda), \sigma(A) \neq \emptyset\)
\end{enumerate}

\begin{definition}{Compact operator}{}
    \(A \in \mathcal{L}(H)\) is compact \(\iff AB_{1}(0)\) is a precompact set in \(H\).
\end{definition}

\begin{definition}{Adjoint operator}{}
    Let \(A \in  \mathcal{L}(H)\). The adjoint operator \(A^{*} \in \mathcal{L}(H)\) is defined via \((Ax, y) = (x, A^{*}y), \quad \forall x, y \in H\).
    It exists due to Riesz representation theorem. In the finite dimensional case, the adjoint operator coincides with the transpose operator.
\end{definition}

\begin{definition}{Self-adjoint operator}{}
    \(A \in \mathcal{L}(H)\) is self-adjoint if \(A = A^{*}\) 
\end{definition}

\begin{definition}{Fredholm}{}
    \(A \in \mathcal{L}(H)\) is Fredholm if Range(\(A\)) and Range(\(A^{*}\)) are closed and \(\ker(A)\) and \(\ker(A^{*})\) are both finite dimensional.

    Then the index of \(A\) is defined by 
    \[
        \text{ind}(A) \coloneqq \dim \ker(A) - \dim \ker(A^{*})
    \]
\end{definition}

\begin{theorem}{Key theorem of Fredholm operators theorem}{}
    \(\text{ind}(A)\) is a topological invariant. Namely, if \(A(t), t \in [0, 1]\) is a continuous curve of Fredholm operators, then 
    \[
        \text{ind}(A(0)) = \text{ind}(A(1)).
    \]
\end{theorem}

\begin{definition}{Essential spectrum}{}
    \(\lambda \in \sigma_{\text{ess}}(A)\) if \(\lambda I - A\) is not Fredholm.
\end{definition}

Properties:
\begin{enumerate}
    \item \(K\) is compact \(\iff K^{*}\) is compact.
    \item \(A \in \mathcal{L}(H), K \) compact \(\implies AK\) and \(KA\) are compact.
    \item \(A\) is Fredholm \(\iff A^{*}\) is Fredholm.
    \item Fredholm alternative: Let \(A\) be Fredholm. Then:
    \begin{align*}
        H &= \text{Range}(A) \oplus \ker(A^{*}) \\
        H &= \text{Range}(A^{*}) \oplus \ker(A)
    \end{align*}
    \item \(A\) is Fredholm \(\iff\) it is invertible by modulus of compact operators, i.e. \(\exists B : AB = I + K_{1}, BA = I + K_{2}\), with \(K_1, K_2\) compact.
    \item Let \(K\) be a compact operator. Then \(\sigma_{\text{ess}}(K) = 0\) and for any \(\varepsilon > 0, \sigma(K) \setminus B_{|varepsilon}(0)\) consists of finitely many eigenvalues of finite multiplicity.
    \item If \(A = A^{*}\), then \(\sigma(A)\) is real and \(\sigma_{R}(A) = \emptyset\).
\end{enumerate}
Thus, the simplest case is the case of positive, compact and self-adjoint operators.

\begin{theorem}{Hilbert-Schmidt}{}
    Let \(A \in \mathcal{L}(H)\) be a compact, self-adjoint and positive (\((Ax, x) > 0\) if \(x \neq 0\)) operator. Then there exists a sequence of non-zero real eigenvalues \(\lambda_{i} \in \sigma_{p}(A)\) such that \(\vert \lambda_{i} \vert \) is monotonically non-increasing
    \[
        \lambda_{1} \geq \lambda_{2} \geq \lambda_{3} \geq \dots,
    \]
    and the corresponding eigenvectors \(\{e_{n}\}_{n=1}^{\infty} (Ae_{n} = \lambda e_{n})\) form the orthonormal basis in \(H\).

    Moreover, any \(x \in H\) can be written as
    \[
        x = \sum_{n=1}^{\infty} \left\langle x, e_{n} \right\rangle e_{n},
    \]
    and \(A\) can be written as
    \[
        Ax = \sum_{n=1}^{\infty} \lambda_{n} \left\langle x, e_{n} \right\rangle e_{n}.
    \]
\end{theorem}

\subsection*{Applications}
\begin{example}{Spectrum of the Laplacian with Dirichlet boundary conditions}{}
    Let \(\Omega \in \mathbb{R}^{n}\) be a bounded domain with \(\partial \Omega\) smooth. Consider the Laplace equation with Dirichlet boundary conditions:
    \begin{equation}\label{eq:3.15}
        \begin{cases}
            - \Delta u = f, \qquad f \in L^{2}(\Omega) \\
            \left. u \right|_{\partial \Omega} = 0
        \end{cases}
    \end{equation}

    \(- \Delta\) is not a bounded operator in \(H \coloneqq L^{2}(\Omega)\), so we cannot apply directly apply the Hilbert-Schmidt theorem.

    Let us consider the inverse operator \(A = (- \Delta)^{-1}\) constructed via weak solutions, namely \(u = (- \Delta)^{-1}f = Af\) solves \([u, \varphi] = (f, \varphi)\). That is, \(\forall \varphi \in C^{\infty}_{0}(\Omega)\) (or \(H^{1}_{0}\)) and \(u \in H^{1}_{0}(\Omega)\), we have
    \[
        \int_{\Omega} \nabla u \nabla \varphi \,\mathrm{d}x = \int_{\Omega} f \varphi \,\mathrm{d}x
    \]

    \begin{enumerate}
        \item \(A\) is a bounded operator from \(H\) to \(H^{1}_{0}(\Omega)\). Indeed, let \(\varphi = u\). Then
        \begin{align*}
            \|u\|_{H^{1}_{0}}^{2} &= (f, u) \\
            &\leq \|f\|_{L^{2}}\|u\|_{L^{2}} \\
            &\leq C\|f\|_{L^{2}}\|u\|_{H^{1}_{0}}
        \end{align*}
        Hence 
        \[
            \|u\|_{H^{1}_{0}} \leq C \|f\|_{L^{2}} \implies \frac{\|Af\|_{H^{1}_{0}}}{\|f\|_{L^{2}}} \leq C
        \]
        \item \(A\) is compact since the embedding \(H^{1}_{0} \subset H\) is compact.
        \item \(A\) is self-adjoint. Let \(f, g \in L^{2}(\Omega), u = (- \Delta)^{-1}f, v = (- \Delta)^{-1}g\). Take \(\varphi = v\) in variational formulation for \(u\), and \(\varphi = u\) in variational formulation for \(v\):
        \begin{align*}
            [u, v] &= (f, v) \\
            [u, v] &= (g, u) \\
            \implies (f, v) &= (g, u) \implies (f, Ag) = (g, A^{*}f) \implies A = A^{*}
        \end{align*}
        \item \(A\) is positive.
        \[
            0 < [u, u] = (f, u) = (f, Af)
        \]    
    \end{enumerate}
    By the Hilbert-Schmidt theorem, there exists a complete orthonormal system \(\{e_{n}\}\) of eigenvectors of \(A\) with \(Ae_{n} = \lambda e_{n}\). 
        
    \(e_{n}\) by definition solves \(\lambda_{n}[e_{n}, \varphi_{n}] = (e_{n}, \varphi)\). Indeed, for \(u_{n} = Ae_{n}\), we have \([u_{n}, \varphi] = (e_{n}, \varphi)\). Since \(u_{n} = \lambda_{n}e_{n}\), this implies that \([e_{n}, \varphi] = \lambda_{n}^{-1}(e_{n}, \varphi)\). Hence \(e_{n}\) is a weak solution of \ref{eq:3.15}.

\end{example}

\end{document}
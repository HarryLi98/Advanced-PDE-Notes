\documentclass{report}

%%%%%%%%%%%%%%%%%%%%%%%%%%%%%%%%%
% PACKAGE IMPORTS
%%%%%%%%%%%%%%%%%%%%%%%%%%%%%%%%%


\usepackage[tmargin=2cm,rmargin=1in,lmargin=1in,margin=0.85in,bmargin=2cm,footskip=.2in]{geometry}
\usepackage{amsmath,amsfonts,amsthm,amssymb,mathtools}
\usepackage[varbb]{newpxmath}
\usepackage{xfrac}
\usepackage[makeroom]{cancel}
\usepackage{mathtools}
\usepackage{bookmark}
\usepackage{enumitem}
\usepackage{hyperref,theoremref}
\hypersetup{
	pdftitle={Assignment},
	colorlinks=true, linkcolor=doc!90,
	bookmarksnumbered=true,
	bookmarksopen=true
}
\usepackage[most,many,breakable]{tcolorbox}
\usepackage{xcolor}
\usepackage{varwidth}
\usepackage{varwidth}
\usepackage{etoolbox}
%\usepackage{authblk}
\usepackage{nameref}
\usepackage{multicol,array}
\usepackage{tikz-cd}
\usepackage[ruled,vlined,linesnumbered]{algorithm2e}
\usepackage{comment} % enables the use of multi-line comments (\ifx \fi) 
\usepackage{import}
\usepackage{xifthen}
\usepackage{pdfpages}
\usepackage{transparent}

\newcommand\mycommfont[1]{\footnotesize\ttfamily\textcolor{blue}{#1}}
\SetCommentSty{mycommfont}
\newcommand{\incfig}[1]{%
    \def\svgwidth{\columnwidth}
    \import{./figures/}{#1.pdf_tex}
}

\usepackage{tikzsymbols}
\renewcommand\qedsymbol{$\Laughey$}


%\usepackage{import}
%\usepackage{xifthen}
%\usepackage{pdfpages}
%\usepackage{transparent}


%%%%%%%%%%%%%%%%%%%%%%%%%%%%%%
% SELF MADE COLORS
%%%%%%%%%%%%%%%%%%%%%%%%%%%%%%



\definecolor{myg}{RGB}{56, 140, 70}
\definecolor{myb}{RGB}{45, 111, 177}
\definecolor{myr}{RGB}{199, 68, 64}
\definecolor{mytheorembg}{HTML}{F2F2F9}
\definecolor{mytheoremfr}{HTML}{00007B}
\definecolor{mylemmabg}{HTML}{FFFAF8}
\definecolor{mylemmafr}{HTML}{983b0f}
\definecolor{mypropbg}{HTML}{f2fbfc}
\definecolor{mypropfr}{HTML}{191971}
\definecolor{myexamplebg}{HTML}{F2FBF8}
\definecolor{myexamplefr}{HTML}{88D6D1}
\definecolor{myexampleti}{HTML}{2A7F7F}
\definecolor{mydefinitbg}{HTML}{E5E5FF}
\definecolor{mydefinitfr}{HTML}{3F3FA3}
\definecolor{notesgreen}{RGB}{0,162,0}
\definecolor{myp}{RGB}{197, 92, 212}
\definecolor{mygr}{HTML}{2C3338}
\definecolor{myred}{RGB}{127,0,0}
\definecolor{myyellow}{RGB}{169,121,69}
\definecolor{myexercisebg}{HTML}{F2FBF8}
\definecolor{myexercisefg}{HTML}{88D6D1}


%%%%%%%%%%%%%%%%%%%%%%%%%%%%
% TCOLORBOX SETUPS
%%%%%%%%%%%%%%%%%%%%%%%%%%%%

\setlength{\parindent}{1cm}
%================================
% THEOREM BOX
%================================

\tcbuselibrary{theorems,skins,hooks}
\newtcbtheorem[number within=section]{Theorem}{Theorem}
{%
	enhanced,
	breakable,
	colback = mytheorembg,
	frame hidden,
	boxrule = 0sp,
	borderline west = {2pt}{0pt}{mytheoremfr},
	sharp corners,
	detach title,
	before upper = \tcbtitle\par\smallskip,
	coltitle = mytheoremfr,
	fonttitle = \bfseries\sffamily,
	description font = \mdseries,
	separator sign none,
	segmentation style={solid, mytheoremfr},
}
{th}

\tcbuselibrary{theorems,skins,hooks}
\newtcbtheorem[number within=chapter]{theorem}{Theorem}
{%
	enhanced,
	breakable,
	colback = mytheorembg,
	frame hidden,
	boxrule = 0sp,
	borderline west = {2pt}{0pt}{mytheoremfr},
	sharp corners,
	detach title,
	before upper = \tcbtitle\par\smallskip,
	coltitle = mytheoremfr,
	fonttitle = \bfseries\sffamily,
	description font = \mdseries,
	separator sign none,
	segmentation style={solid, mytheoremfr},
}
{th}


\tcbuselibrary{theorems,skins,hooks}
\newtcolorbox{Theoremcon}
{%
	enhanced
	,breakable
	,colback = mytheorembg
	,frame hidden
	,boxrule = 0sp
	,borderline west = {2pt}{0pt}{mytheoremfr}
	,sharp corners
	,description font = \mdseries
	,separator sign none
}

%================================
% Corollery
%================================
\tcbuselibrary{theorems,skins,hooks}
\newtcbtheorem[number within=section]{Corollary}{Corollary}
{%
	enhanced
	,breakable
	,colback = myp!10
	,frame hidden
	,boxrule = 0sp
	,borderline west = {2pt}{0pt}{myp!85!black}
	,sharp corners
	,detach title
	,before upper = \tcbtitle\par\smallskip
	,coltitle = myp!85!black
	,fonttitle = \bfseries\sffamily
	,description font = \mdseries
	,separator sign none
	,segmentation style={solid, myp!85!black}
}
{th}
\tcbuselibrary{theorems,skins,hooks}
\newtcbtheorem[number within=chapter]{corollary}{Corollary}
{%
	enhanced
	,breakable
	,colback = myp!10
	,frame hidden
	,boxrule = 0sp
	,borderline west = {2pt}{0pt}{myp!85!black}
	,sharp corners
	,detach title
	,before upper = \tcbtitle\par\smallskip
	,coltitle = myp!85!black
	,fonttitle = \bfseries\sffamily
	,description font = \mdseries
	,separator sign none
	,segmentation style={solid, myp!85!black}
}
{th}


%================================
% lemma
%================================

\tcbuselibrary{theorems,skins,hooks}
\newtcbtheorem[number within=section]{lemma}{lemma}
{%
	enhanced,
	breakable,
	colback = mylemmabg,
	frame hidden,
	boxrule = 0sp,
	borderline west = {2pt}{0pt}{mylemmafr},
	sharp corners,
	detach title,
	before upper = \tcbtitle\par\smallskip,
	coltitle = mylemmafr,
	fonttitle = \bfseries\sffamily,
	description font = \mdseries,
	separator sign none,
	segmentation style={solid, mylemmafr},
}
{th}

%================================
% PROPOSITION
%================================

\tcbuselibrary{theorems,skins,hooks}
\newtcbtheorem[number within=section]{Prop}{Proposition}
{%
	enhanced,
	breakable,
	colback = mypropbg,
	frame hidden,
	boxrule = 0sp,
	borderline west = {2pt}{0pt}{mypropfr},
	sharp corners,
	detach title,
	before upper = \tcbtitle\par\smallskip,
	coltitle = mypropfr,
	fonttitle = \bfseries\sffamily,
	description font = \mdseries,
	separator sign none,
	segmentation style={solid, mypropfr},
}
{th}

\tcbuselibrary{theorems,skins,hooks}
\newtcbtheorem[number within=chapter]{prop}{Proposition}
{%
	enhanced,
	breakable,
	colback = mypropbg,
	frame hidden,
	boxrule = 0sp,
	borderline west = {2pt}{0pt}{mypropfr},
	sharp corners,
	detach title,
	before upper = \tcbtitle\par\smallskip,
	coltitle = mypropfr,
	fonttitle = \bfseries\sffamily,
	description font = \mdseries,
	separator sign none,
	segmentation style={solid, mypropfr},
}
{th}


%================================
% CLAIM
%================================

\tcbuselibrary{theorems,skins,hooks}
\newtcbtheorem[number within=section]{claim}{Claim}
{%
	enhanced
	,breakable
	,colback = myg!10
	,frame hidden
	,boxrule = 0sp
	,borderline west = {2pt}{0pt}{myg}
	,sharp corners
	,detach title
	,before upper = \tcbtitle\par\smallskip
	,coltitle = myg!85!black
	,fonttitle = \bfseries\sffamily
	,description font = \mdseries
	,separator sign none
	,segmentation style={solid, myg!85!black}
}
{th}



%================================
% Exercise
%================================

\tcbuselibrary{theorems,skins,hooks}
\newtcbtheorem[number within=section]{Exercise}{Exercise}
{%
	enhanced,
	breakable,
	colback = myexercisebg,
	frame hidden,
	boxrule = 0sp,
	borderline west = {2pt}{0pt}{myexercisefg},
	sharp corners,
	detach title,
	before upper = \tcbtitle\par\smallskip,
	coltitle = myexercisefg,
	fonttitle = \bfseries\sffamily,
	description font = \mdseries,
	separator sign none,
	segmentation style={solid, myexercisefg},
}
{th}

\tcbuselibrary{theorems,skins,hooks}
\newtcbtheorem[number within=chapter]{exercise}{Exercise}
{%
	enhanced,
	breakable,
	colback = myexercisebg,
	frame hidden,
	boxrule = 0sp,
	borderline west = {2pt}{0pt}{myexercisefg},
	sharp corners,
	detach title,
	before upper = \tcbtitle\par\smallskip,
	coltitle = myexercisefg,
	fonttitle = \bfseries\sffamily,
	description font = \mdseries,
	separator sign none,
	segmentation style={solid, myexercisefg},
}
{th}

%================================
% EXAMPLE BOX
%================================

\newtcbtheorem[number within=section]{Example}{Example}
{%
	colback = myexamplebg
	,breakable
	,colframe = myexamplefr
	,coltitle = myexampleti
	,boxrule = 1pt
	,sharp corners
	,detach title
	,before upper=\tcbtitle\par\smallskip
	,fonttitle = \bfseries
	,description font = \mdseries
	,separator sign none
	,description delimiters parenthesis
}
{ex}

\newtcbtheorem[number within=chapter]{example}{Example}
{%
	colback = myexamplebg
	,breakable
	,colframe = myexamplefr
	,coltitle = myexampleti
	,boxrule = 1pt
	,sharp corners
	,detach title
	,before upper=\tcbtitle\par\smallskip
	,fonttitle = \bfseries
	,description font = \mdseries
	,separator sign none
	,description delimiters parenthesis
}
{ex}

%================================
% DEFINITION BOX
%================================

\newtcbtheorem[number within=section]{Definition}{Definition}{enhanced,
	before skip=2mm,after skip=2mm, colback=red!5,colframe=red!80!black,boxrule=0.5mm,
	attach boxed title to top left={xshift=1cm,yshift*=1mm-\tcboxedtitleheight}, varwidth boxed title*=-3cm,
	boxed title style={frame code={
					\path[fill=tcbcolback]
					([yshift=-1mm,xshift=-1mm]frame.north west)
					arc[start angle=0,end angle=180,radius=1mm]
					([yshift=-1mm,xshift=1mm]frame.north east)
					arc[start angle=180,end angle=0,radius=1mm];
					\path[left color=tcbcolback!60!black,right color=tcbcolback!60!black,
						middle color=tcbcolback!80!black]
					([xshift=-2mm]frame.north west) -- ([xshift=2mm]frame.north east)
					[rounded corners=1mm]-- ([xshift=1mm,yshift=-1mm]frame.north east)
					-- (frame.south east) -- (frame.south west)
					-- ([xshift=-1mm,yshift=-1mm]frame.north west)
					[sharp corners]-- cycle;
				},interior engine=empty,
		},
	fonttitle=\bfseries,
	title={#2},#1}{def}
\newtcbtheorem[number within=chapter]{definition}{Definition}{enhanced,
	before skip=2mm,after skip=2mm, colback=red!5,colframe=red!80!black,boxrule=0.5mm,
	attach boxed title to top left={xshift=1cm,yshift*=1mm-\tcboxedtitleheight}, varwidth boxed title*=-3cm,
	boxed title style={frame code={
					\path[fill=tcbcolback]
					([yshift=-1mm,xshift=-1mm]frame.north west)
					arc[start angle=0,end angle=180,radius=1mm]
					([yshift=-1mm,xshift=1mm]frame.north east)
					arc[start angle=180,end angle=0,radius=1mm];
					\path[left color=tcbcolback!60!black,right color=tcbcolback!60!black,
						middle color=tcbcolback!80!black]
					([xshift=-2mm]frame.north west) -- ([xshift=2mm]frame.north east)
					[rounded corners=1mm]-- ([xshift=1mm,yshift=-1mm]frame.north east)
					-- (frame.south east) -- (frame.south west)
					-- ([xshift=-1mm,yshift=-1mm]frame.north west)
					[sharp corners]-- cycle;
				},interior engine=empty,
		},
	fonttitle=\bfseries,
	title={#2},#1}{def}



%================================
% Solution BOX
%================================

\makeatletter
\newtcbtheorem{question}{Question}{enhanced,
	breakable,
	colback=white,
	colframe=myb!80!black,
	attach boxed title to top left={yshift*=-\tcboxedtitleheight},
	fonttitle=\bfseries,
	title={#2},
	boxed title size=title,
	boxed title style={%
			sharp corners,
			rounded corners=northwest,
			colback=tcbcolframe,
			boxrule=0pt,
		},
	underlay boxed title={%
			\path[fill=tcbcolframe] (title.south west)--(title.south east)
			to[out=0, in=180] ([xshift=5mm]title.east)--
			(title.center-|frame.east)
			[rounded corners=\kvtcb@arc] |-
			(frame.north) -| cycle;
		},
	#1
}{def}
\makeatother

%================================
% SOLUTION BOX
%================================

\makeatletter
\newtcolorbox{solution}{enhanced,
	breakable,
	colback=white,
	colframe=myg!80!black,
	attach boxed title to top left={yshift*=-\tcboxedtitleheight},
	title=Solution,
	boxed title size=title,
	boxed title style={%
			sharp corners,
			rounded corners=northwest,
			colback=tcbcolframe,
			boxrule=0pt,
		},
	underlay boxed title={%
			\path[fill=tcbcolframe] (title.south west)--(title.south east)
			to[out=0, in=180] ([xshift=5mm]title.east)--
			(title.center-|frame.east)
			[rounded corners=\kvtcb@arc] |-
			(frame.north) -| cycle;
		},
}
\makeatother

%================================
% Question BOX
%================================

\makeatletter
\newtcbtheorem{qstion}{Question}{enhanced,
	breakable,
	colback=white,
	colframe=mygr,
	attach boxed title to top left={yshift*=-\tcboxedtitleheight},
	fonttitle=\bfseries,
	title={#2},
	boxed title size=title,
	boxed title style={%
			sharp corners,
			rounded corners=northwest,
			colback=tcbcolframe,
			boxrule=0pt,
		},
	underlay boxed title={%
			\path[fill=tcbcolframe] (title.south west)--(title.south east)
			to[out=0, in=180] ([xshift=5mm]title.east)--
			(title.center-|frame.east)
			[rounded corners=\kvtcb@arc] |-
			(frame.north) -| cycle;
		},
	#1
}{def}
\makeatother

\newtcbtheorem[number within=chapter]{wconc}{Wrong Concept}{
	breakable,
	enhanced,
	colback=white,
	colframe=myr,
	arc=0pt,
	outer arc=0pt,
	fonttitle=\bfseries\sffamily\large,
	colbacktitle=myr,
	attach boxed title to top left={},
	boxed title style={
			enhanced,
			skin=enhancedfirst jigsaw,
			arc=3pt,
			bottom=0pt,
			interior style={fill=myr}
		},
	#1
}{def}



%================================
% NOTE BOX
%================================

\usetikzlibrary{arrows,calc,shadows.blur}
\tcbuselibrary{skins}
\newtcolorbox{note}[1][]{%
	enhanced jigsaw,
	colback=gray!20!white,%
	colframe=gray!80!black,
	size=small,
	boxrule=1pt,
	title=\textbf{Note:-},
	halign title=flush center,
	coltitle=black,
	breakable,
	drop shadow=black!50!white,
	attach boxed title to top left={xshift=1cm,yshift=-\tcboxedtitleheight/2,yshifttext=-\tcboxedtitleheight/2},
	minipage boxed title=1.5cm,
	boxed title style={%
			colback=white,
			size=fbox,
			boxrule=1pt,
			boxsep=2pt,
			underlay={%
					\coordinate (dotA) at ($(interior.west) + (-0.5pt,0)$);
					\coordinate (dotB) at ($(interior.east) + (0.5pt,0)$);
					\begin{scope}
						\clip (interior.north west) rectangle ([xshift=3ex]interior.east);
						\filldraw [white, blur shadow={shadow opacity=60, shadow yshift=-.75ex}, rounded corners=2pt] (interior.north west) rectangle (interior.south east);
					\end{scope}
					\begin{scope}[gray!80!black]
						\fill (dotA) circle (2pt);
						\fill (dotB) circle (2pt);
					\end{scope}
				},
		},
	#1,
}

%%%%%%%%%%%%%%%%%%%%%%%%%%%%%%
% SELF MADE COMMANDS
%%%%%%%%%%%%%%%%%%%%%%%%%%%%%%


\newcommand{\thm}[2]{\begin{Theorem}{#1}{}#2\end{Theorem}}
\newcommand{\cor}[2]{\begin{Corollary}{#1}{}#2\end{Corollary}}
\newcommand{\mlemma}[2]{\begin{lemma}{#1}{}#2\end{lemma}}
\newcommand{\mprop}[2]{\begin{Prop}{#1}{}#2\end{Prop}}
\newcommand{\clm}[3]{\begin{claim}{#1}{#2}#3\end{claim}}
\newcommand{\wc}[2]{\begin{wconc}{#1}{}\setlength{\parindent}{1cm}#2\end{wconc}}
\newcommand{\thmcon}[1]{\begin{Theoremcon}{#1}\end{Theoremcon}}
\newcommand{\ex}[2]{\begin{Example}{#1}{}#2\end{Example}}
\newcommand{\dfn}[2]{\begin{Definition}[colbacktitle=red!75!black]{#1}{}#2\end{Definition}}
\newcommand{\dfnc}[2]{\begin{definition}[colbacktitle=red!75!black]{#1}{}#2\end{definition}}
\newcommand{\qs}[2]{\begin{question}{#1}{}#2\end{question}}
\newcommand{\pf}[2]{\begin{myproof}[#1]#2\end{myproof}}
\newcommand{\nt}[1]{\begin{note}#1\end{note}}

\newcommand*\circled[1]{\tikz[baseline=(char.base)]{
		\node[shape=circle,draw,inner sep=1pt] (char) {#1};}}
\newcommand\getcurrentref[1]{%
	\ifnumequal{\value{#1}}{0}
	{??}
	{\the\value{#1}}%
}
\newcommand{\getCurrentSectionNumber}{\getcurrentref{section}}
\newenvironment{myproof}[1][\proofname]{%
	\proof[\bfseries #1: ]%
}{\endproof}

\newcommand{\mclm}[2]{\begin{myclaim}[#1]#2\end{myclaim}}
\newenvironment{myclaim}[1][\claimname]{\proof[\bfseries #1: ]}{}

\newcounter{mylabelcounter}

\makeatletter
\newcommand{\setword}[2]{%
	\phantomsection
	#1\def\@currentlabel{\unexpanded{#1}}\label{#2}%
}
\makeatother




\tikzset{
	symbol/.style={
			draw=none,
			every to/.append style={
					edge node={node [sloped, allow upside down, auto=false]{$#1$}}}
		}
}


% deliminators
\DeclarePairedDelimiter{\abs}{\lvert}{\rvert}
\DeclarePairedDelimiter{\norm}{\lVert}{\rVert}

\DeclarePairedDelimiter{\ceil}{\lceil}{\rceil}
\DeclarePairedDelimiter{\floor}{\lfloor}{\rfloor}
\DeclarePairedDelimiter{\round}{\lfloor}{\rceil}

\newsavebox\diffdbox
\newcommand{\slantedromand}{{\mathpalette\makesl{d}}}
\newcommand{\makesl}[2]{%
\begingroup
\sbox{\diffdbox}{$\mathsurround=0pt#1\mathrm{#2}$}%
\pdfsave
\pdfsetmatrix{1 0 0.2 1}%
\rlap{\usebox{\diffdbox}}%
\pdfrestore
\hskip\wd\diffdbox
\endgroup
}
\newcommand{\dd}[1][]{\ensuremath{\mathop{}\!\ifstrempty{#1}{%
\slantedromand\@ifnextchar^{\hspace{0.2ex}}{\hspace{0.1ex}}}%
{\slantedromand\hspace{0.2ex}^{#1}}}}
\ProvideDocumentCommand\dv{o m g}{%
  \ensuremath{%
    \IfValueTF{#3}{%
      \IfNoValueTF{#1}{%
        \frac{\dd #2}{\dd #3}%
      }{%
        \frac{\dd^{#1} #2}{\dd #3^{#1}}%
      }%
    }{%
      \IfNoValueTF{#1}{%
        \frac{\dd}{\dd #2}%
      }{%
        \frac{\dd^{#1}}{\dd #2^{#1}}%
      }%
    }%
  }%
}
\providecommand*{\pdv}[3][]{\frac{\partial^{#1}#2}{\partial#3^{#1}}}
%  - others
\DeclareMathOperator{\Lap}{\mathcal{L}}
\DeclareMathOperator{\Var}{Var} % varience
\DeclareMathOperator{\Cov}{Cov} % covarience
\DeclareMathOperator{\E}{E} % expected

% Since the amsthm package isn't loaded

% I prefer the slanted \leq
\let\oldleq\leq % save them in case they're every wanted
\let\oldgeq\geq
\renewcommand{\leq}{\leqslant}
\renewcommand{\geq}{\geqslant}

% % redefine matrix env to allow for alignment, use r as default
% \renewcommand*\env@matrix[1][r]{\hskip -\arraycolsep
%     \let\@ifnextchar\new@ifnextchar
%     \array{*\c@MaxMatrixCols #1}}


%\usepackage{framed}
%\usepackage{titletoc}
%\usepackage{etoolbox}
%\usepackage{lmodern}


%\patchcmd{\tableofcontents}{\contentsname}{\sffamily\contentsname}{}{}

%\renewenvironment{leftbar}
%{\def\FrameCommand{\hspace{6em}%
%		{\color{myyellow}\vrule width 2pt depth 6pt}\hspace{1em}}%
%	\MakeFramed{\parshape 1 0cm \dimexpr\textwidth-6em\relax\FrameRestore}\vskip2pt%
%}
%{\endMakeFramed}

%\titlecontents{chapter}
%[0em]{\vspace*{2\baselineskip}}
%{\parbox{4.5em}{%
%		\hfill\Huge\sffamily\bfseries\color{myred}\thecontentspage}%
%	\vspace*{-2.3\baselineskip}\leftbar\textsc{\small\chaptername~\thecontentslabel}\\\sffamily}
%{}{\endleftbar}
%\titlecontents{section}
%[8.4em]
%{\sffamily\contentslabel{3em}}{}{}
%{\hspace{0.5em}\nobreak\itshape\color{myred}\contentspage}
%\titlecontents{subsection}
%[8.4em]
%{\sffamily\contentslabel{3em}}{}{}  
%{\hspace{0.5em}\nobreak\itshape\color{myred}\contentspage}



%%%%%%%%%%%%%%%%%%%%%%%%%%%%%%%%%%%%%%%%%%%
% TABLE OF CONTENTS
%%%%%%%%%%%%%%%%%%%%%%%%%%%%%%%%%%%%%%%%%%%

\usepackage{tikz}
\definecolor{doc}{RGB}{0,60,110}
\usepackage{titletoc}
\contentsmargin{0cm}
\titlecontents{chapter}[3.7pc]
{\addvspace{30pt}%
	\begin{tikzpicture}[remember picture, overlay]%
		\draw[fill=doc!60,draw=doc!60] (-7,-.1) rectangle (-0.9,.5);%
		\pgftext[left,x=-3.5cm,y=0.2cm]{\color{white}\Large\sc\bfseries Chapter\ \thecontentslabel};%
	\end{tikzpicture}\color{doc!60}\large\sc\bfseries}%
{}
{}
{\;\titlerule\;\large\sc\bfseries Page \thecontentspage
	\begin{tikzpicture}[remember picture, overlay]
		\draw[fill=doc!60,draw=doc!60] (2pt,0) rectangle (4,0.1pt);
	\end{tikzpicture}}%
\titlecontents{section}[3.7pc]
{\addvspace{2pt}}
{\contentslabel[\thecontentslabel]{2pc}}
{}
{\hfill\small \thecontentspage}
[]
\titlecontents*{subsection}[3.7pc]
{\addvspace{-1pt}\small}
{}
{}
{\ --- \small\thecontentspage}
[ \textbullet\ ][]

\makeatletter
\renewcommand{\tableofcontents}{%
	\chapter*{%
	  \vspace*{-20\p@}%
	  \begin{tikzpicture}[remember picture, overlay]%
		  \pgftext[right,x=15cm,y=0.2cm]{\color{doc!60}\Huge\sc\bfseries \contentsname};%
		  \draw[fill=doc!60,draw=doc!60] (13,-.75) rectangle (20,1);%
		  \clip (13,-.75) rectangle (20,1);
		  \pgftext[right,x=15cm,y=0.2cm]{\color{white}\Huge\sc\bfseries \contentsname};%
	  \end{tikzpicture}}%
	\@starttoc{toc}}
\makeatother

%From M275 "Topology" at SJSU
\newcommand{\id}{\mathrm{id}}
\newcommand{\taking}[1]{\xrightarrow{#1}}
\newcommand{\inv}{^{-1}}

%From M170 "Introduction to Graph Theory" at SJSU
\DeclareMathOperator{\diam}{diam}
\DeclareMathOperator{\ord}{ord}
\newcommand{\defeq}{\overset{\mathrm{def}}{=}}

%From the USAMO .tex files
\newcommand{\ts}{\textsuperscript}
\newcommand{\dg}{^\circ}
\newcommand{\ii}{\item}

% % From Math 55 and Math 145 at Harvard
% \newenvironment{subproof}[1][Proof]{%
% \begin{proof}[#1] \renewcommand{\qedsymbol}{$\blacksquare$}}%
% {\end{proof}}

\newcommand{\liff}{\leftrightarrow}
\newcommand{\lthen}{\rightarrow}
\newcommand{\opname}{\operatorname}
\newcommand{\surjto}{\twoheadrightarrow}
\newcommand{\injto}{\hookrightarrow}
\newcommand{\On}{\mathrm{On}} % ordinals
\DeclareMathOperator{\img}{im} % Image
\DeclareMathOperator{\Img}{Im} % Image
\DeclareMathOperator{\coker}{coker} % Cokernel
\DeclareMathOperator{\Coker}{Coker} % Cokernel
\DeclareMathOperator{\Ker}{Ker} % Kernel
\DeclareMathOperator{\rank}{rank}
\DeclareMathOperator{\Spec}{Spec} % spectrum
\DeclareMathOperator{\Tr}{Tr} % trace
\DeclareMathOperator{\pr}{pr} % projection
\DeclareMathOperator{\ext}{ext} % extension
\DeclareMathOperator{\pred}{pred} % predecessor
\DeclareMathOperator{\dom}{dom} % domain
\DeclareMathOperator{\ran}{ran} % range
\DeclareMathOperator{\Hom}{Hom} % homomorphism
\DeclareMathOperator{\Mor}{Mor} % morphisms
\DeclareMathOperator{\End}{End} % endomorphism

\newcommand{\eps}{\epsilon}
\newcommand{\veps}{\varepsilon}
\newcommand{\ol}{\overline}
\newcommand{\ul}{\underline}
\newcommand{\wt}{\widetilde}
\newcommand{\wh}{\widehat}
\newcommand{\vocab}[1]{\textbf{\color{blue} #1}}
\providecommand{\half}{\frac{1}{2}}
\newcommand{\dang}{\measuredangle} %% Directed angle
\newcommand{\ray}[1]{\overrightarrow{#1}}
\newcommand{\seg}[1]{\overline{#1}}
\newcommand{\arc}[1]{\wideparen{#1}}
\DeclareMathOperator{\cis}{cis}
\DeclareMathOperator*{\lcm}{lcm}
\DeclareMathOperator*{\argmin}{arg min}
\DeclareMathOperator*{\argmax}{arg max}
\newcommand{\cycsum}{\sum_{\mathrm{cyc}}}
\newcommand{\symsum}{\sum_{\mathrm{sym}}}
\newcommand{\cycprod}{\prod_{\mathrm{cyc}}}
\newcommand{\symprod}{\prod_{\mathrm{sym}}}
\newcommand{\Qed}{\begin{flushright}\qed\end{flushright}}
\newcommand{\parinn}{\setlength{\parindent}{1cm}}
\newcommand{\parinf}{\setlength{\parindent}{0cm}}
% \newcommand{\norm}{\|\cdot\|}
\newcommand{\inorm}{\norm_{\infty}}
\newcommand{\opensets}{\{V_{\alpha}\}_{\alpha\in I}}
\newcommand{\oset}{V_{\alpha}}
\newcommand{\opset}[1]{V_{\alpha_{#1}}}
\newcommand{\lub}{\text{lub}}
\newcommand{\del}[2]{\frac{\partial #1}{\partial #2}}
\newcommand{\Del}[3]{\frac{\partial^{#1} #2}{\partial^{#1} #3}}
\newcommand{\deld}[2]{\dfrac{\partial #1}{\partial #2}}
\newcommand{\Deld}[3]{\dfrac{\partial^{#1} #2}{\partial^{#1} #3}}
\newcommand{\lm}{\lambda}
\newcommand{\uin}{\mathbin{\rotatebox[origin=c]{90}{$\in$}}}
\newcommand{\usubset}{\mathbin{\rotatebox[origin=c]{90}{$\subset$}}}
\newcommand{\lt}{\left}
\newcommand{\rt}{\right}
\newcommand{\bs}[1]{\boldsymbol{#1}}
\newcommand{\exs}{\exists}
\newcommand{\st}{\strut}
\newcommand{\dps}[1]{\displaystyle{#1}}

\newcommand{\sol}{\setlength{\parindent}{0cm}\textbf{\textit{Solution:}}\setlength{\parindent}{1cm} }
\newcommand{\solve}[1]{\setlength{\parindent}{0cm}\textbf{\textit{Solution: }}\setlength{\parindent}{1cm}#1 \Qed}
% Things Lie
\newcommand{\kb}{\mathfrak b}
\newcommand{\kg}{\mathfrak g}
\newcommand{\kh}{\mathfrak h}
\newcommand{\kn}{\mathfrak n}
\newcommand{\ku}{\mathfrak u}
\newcommand{\kz}{\mathfrak z}
\DeclareMathOperator{\Ext}{Ext} % Ext functor
\DeclareMathOperator{\Tor}{Tor} % Tor functor
\newcommand{\gl}{\opname{\mathfrak{gl}}} % frak gl group
\renewcommand{\sl}{\opname{\mathfrak{sl}}} % frak sl group chktex 6

% More script letters etc.
\newcommand{\SA}{\mathcal A}
\newcommand{\SB}{\mathcal B}
\newcommand{\SC}{\mathcal C}
\newcommand{\SF}{\mathcal F}
\newcommand{\SG}{\mathcal G}
\newcommand{\SH}{\mathcal H}
\newcommand{\OO}{\mathcal O}

\newcommand{\SCA}{\mathscr A}
\newcommand{\SCB}{\mathscr B}
\newcommand{\SCC}{\mathscr C}
\newcommand{\SCD}{\mathscr D}
\newcommand{\SCE}{\mathscr E}
\newcommand{\SCF}{\mathscr F}
\newcommand{\SCG}{\mathscr G}
\newcommand{\SCH}{\mathscr H}

% Mathfrak primes
\newcommand{\km}{\mathfrak m}
\newcommand{\kp}{\mathfrak p}
\newcommand{\kq}{\mathfrak q}

% number sets
\newcommand{\RR}[1][]{\ensuremath{\ifstrempty{#1}{\mathbb{R}}{\mathbb{R}^{#1}}}}
\newcommand{\NN}[1][]{\ensuremath{\ifstrempty{#1}{\mathbb{N}}{\mathbb{N}^{#1}}}}
\newcommand{\ZZ}[1][]{\ensuremath{\ifstrempty{#1}{\mathbb{Z}}{\mathbb{Z}^{#1}}}}
\newcommand{\QQ}[1][]{\ensuremath{\ifstrempty{#1}{\mathbb{Q}}{\mathbb{Q}^{#1}}}}
\newcommand{\CC}[1][]{\ensuremath{\ifstrempty{#1}{\mathbb{C}}{\mathbb{C}^{#1}}}}
\newcommand{\PP}[1][]{\ensuremath{\ifstrempty{#1}{\mathbb{P}}{\mathbb{P}^{#1}}}}
\newcommand{\HH}[1][]{\ensuremath{\ifstrempty{#1}{\mathbb{H}}{\mathbb{H}^{#1}}}}
\newcommand{\FF}[1][]{\ensuremath{\ifstrempty{#1}{\mathbb{F}}{\mathbb{F}^{#1}}}}
% expected value
\newcommand{\EE}{\ensuremath{\mathbb{E}}}
\newcommand{\charin}{\text{ char }}
\DeclareMathOperator{\sign}{sign}
\DeclareMathOperator{\Aut}{Aut}
\DeclareMathOperator{\Inn}{Inn}
\DeclareMathOperator{\Syl}{Syl}
\DeclareMathOperator{\Gal}{Gal}
\DeclareMathOperator{\GL}{GL} % General linear group
\DeclareMathOperator{\SL}{SL} % Special linear group

%---------------------------------------
% BlackBoard Math Fonts :-
%---------------------------------------

%Captital Letters
\newcommand{\bbA}{\mathbb{A}}	\newcommand{\bbB}{\mathbb{B}}
\newcommand{\bbC}{\mathbb{C}}	\newcommand{\bbD}{\mathbb{D}}
\newcommand{\bbE}{\mathbb{E}}	\newcommand{\bbF}{\mathbb{F}}
\newcommand{\bbG}{\mathbb{G}}	\newcommand{\bbH}{\mathbb{H}}
\newcommand{\bbI}{\mathbb{I}}	\newcommand{\bbJ}{\mathbb{J}}
\newcommand{\bbK}{\mathbb{K}}	\newcommand{\bbL}{\mathbb{L}}
\newcommand{\bbM}{\mathbb{M}}	\newcommand{\bbN}{\mathbb{N}}
\newcommand{\bbO}{\mathbb{O}}	\newcommand{\bbP}{\mathbb{P}}
\newcommand{\bbQ}{\mathbb{Q}}	\newcommand{\bbR}{\mathbb{R}}
\newcommand{\bbS}{\mathbb{S}}	\newcommand{\bbT}{\mathbb{T}}
\newcommand{\bbU}{\mathbb{U}}	\newcommand{\bbV}{\mathbb{V}}
\newcommand{\bbW}{\mathbb{W}}	\newcommand{\bbX}{\mathbb{X}}
\newcommand{\bbY}{\mathbb{Y}}	\newcommand{\bbZ}{\mathbb{Z}}

%---------------------------------------
% MathCal Fonts :-
%---------------------------------------

%Captital Letters
\newcommand{\mcA}{\mathcal{A}}	\newcommand{\mcB}{\mathcal{B}}
\newcommand{\mcC}{\mathcal{C}}	\newcommand{\mcD}{\mathcal{D}}
\newcommand{\mcE}{\mathcal{E}}	\newcommand{\mcF}{\mathcal{F}}
\newcommand{\mcG}{\mathcal{G}}	\newcommand{\mcH}{\mathcal{H}}
\newcommand{\mcI}{\mathcal{I}}	\newcommand{\mcJ}{\mathcal{J}}
\newcommand{\mcK}{\mathcal{K}}	\newcommand{\mcL}{\mathcal{L}}
\newcommand{\mcM}{\mathcal{M}}	\newcommand{\mcN}{\mathcal{N}}
\newcommand{\mcO}{\mathcal{O}}	\newcommand{\mcP}{\mathcal{P}}
\newcommand{\mcQ}{\mathcal{Q}}	\newcommand{\mcR}{\mathcal{R}}
\newcommand{\mcS}{\mathcal{S}}	\newcommand{\mcT}{\mathcal{T}}
\newcommand{\mcU}{\mathcal{U}}	\newcommand{\mcV}{\mathcal{V}}
\newcommand{\mcW}{\mathcal{W}}	\newcommand{\mcX}{\mathcal{X}}
\newcommand{\mcY}{\mathcal{Y}}	\newcommand{\mcZ}{\mathcal{Z}}


%---------------------------------------
% Bold Math Fonts :-
%---------------------------------------

%Captital Letters
\newcommand{\bmA}{\boldsymbol{A}}	\newcommand{\bmB}{\boldsymbol{B}}
\newcommand{\bmC}{\boldsymbol{C}}	\newcommand{\bmD}{\boldsymbol{D}}
\newcommand{\bmE}{\boldsymbol{E}}	\newcommand{\bmF}{\boldsymbol{F}}
\newcommand{\bmG}{\boldsymbol{G}}	\newcommand{\bmH}{\boldsymbol{H}}
\newcommand{\bmI}{\boldsymbol{I}}	\newcommand{\bmJ}{\boldsymbol{J}}
\newcommand{\bmK}{\boldsymbol{K}}	\newcommand{\bmL}{\boldsymbol{L}}
\newcommand{\bmM}{\boldsymbol{M}}	\newcommand{\bmN}{\boldsymbol{N}}
\newcommand{\bmO}{\boldsymbol{O}}	\newcommand{\bmP}{\boldsymbol{P}}
\newcommand{\bmQ}{\boldsymbol{Q}}	\newcommand{\bmR}{\boldsymbol{R}}
\newcommand{\bmS}{\boldsymbol{S}}	\newcommand{\bmT}{\boldsymbol{T}}
\newcommand{\bmU}{\boldsymbol{U}}	\newcommand{\bmV}{\boldsymbol{V}}
\newcommand{\bmW}{\boldsymbol{W}}	\newcommand{\bmX}{\boldsymbol{X}}
\newcommand{\bmY}{\boldsymbol{Y}}	\newcommand{\bmZ}{\boldsymbol{Z}}
%Small Letters
\newcommand{\bma}{\boldsymbol{a}}	\newcommand{\bmb}{\boldsymbol{b}}
\newcommand{\bmc}{\boldsymbol{c}}	\newcommand{\bmd}{\boldsymbol{d}}
\newcommand{\bme}{\boldsymbol{e}}	\newcommand{\bmf}{\boldsymbol{f}}
\newcommand{\bmg}{\boldsymbol{g}}	\newcommand{\bmh}{\boldsymbol{h}}
\newcommand{\bmi}{\boldsymbol{i}}	\newcommand{\bmj}{\boldsymbol{j}}
\newcommand{\bmk}{\boldsymbol{k}}	\newcommand{\bml}{\boldsymbol{l}}
\newcommand{\bmm}{\boldsymbol{m}}	\newcommand{\bmn}{\boldsymbol{n}}
\newcommand{\bmo}{\boldsymbol{o}}	\newcommand{\bmp}{\boldsymbol{p}}
\newcommand{\bmq}{\boldsymbol{q}}	\newcommand{\bmr}{\boldsymbol{r}}
\newcommand{\bms}{\boldsymbol{s}}	\newcommand{\bmt}{\boldsymbol{t}}
\newcommand{\bmu}{\boldsymbol{u}}	\newcommand{\bmv}{\boldsymbol{v}}
\newcommand{\bmw}{\boldsymbol{w}}	\newcommand{\bmx}{\boldsymbol{x}}
\newcommand{\bmy}{\boldsymbol{y}}	\newcommand{\bmz}{\boldsymbol{z}}

%---------------------------------------
% Scr Math Fonts :-
%---------------------------------------

\newcommand{\sA}{{\mathscr{A}}}   \newcommand{\sB}{{\mathscr{B}}}
\newcommand{\sC}{{\mathscr{C}}}   \newcommand{\sD}{{\mathscr{D}}}
\newcommand{\sE}{{\mathscr{E}}}   \newcommand{\sF}{{\mathscr{F}}}
\newcommand{\sG}{{\mathscr{G}}}   \newcommand{\sH}{{\mathscr{H}}}
\newcommand{\sI}{{\mathscr{I}}}   \newcommand{\sJ}{{\mathscr{J}}}
\newcommand{\sK}{{\mathscr{K}}}   \newcommand{\sL}{{\mathscr{L}}}
\newcommand{\sM}{{\mathscr{M}}}   \newcommand{\sN}{{\mathscr{N}}}
\newcommand{\sO}{{\mathscr{O}}}   \newcommand{\sP}{{\mathscr{P}}}
\newcommand{\sQ}{{\mathscr{Q}}}   \newcommand{\sR}{{\mathscr{R}}}
\newcommand{\sS}{{\mathscr{S}}}   \newcommand{\sT}{{\mathscr{T}}}
\newcommand{\sU}{{\mathscr{U}}}   \newcommand{\sV}{{\mathscr{V}}}
\newcommand{\sW}{{\mathscr{W}}}   \newcommand{\sX}{{\mathscr{X}}}
\newcommand{\sY}{{\mathscr{Y}}}   \newcommand{\sZ}{{\mathscr{Z}}}


%---------------------------------------
% Math Fraktur Font
%---------------------------------------

%Captital Letters
\newcommand{\mfA}{\mathfrak{A}}	\newcommand{\mfB}{\mathfrak{B}}
\newcommand{\mfC}{\mathfrak{C}}	\newcommand{\mfD}{\mathfrak{D}}
\newcommand{\mfE}{\mathfrak{E}}	\newcommand{\mfF}{\mathfrak{F}}
\newcommand{\mfG}{\mathfrak{G}}	\newcommand{\mfH}{\mathfrak{H}}
\newcommand{\mfI}{\mathfrak{I}}	\newcommand{\mfJ}{\mathfrak{J}}
\newcommand{\mfK}{\mathfrak{K}}	\newcommand{\mfL}{\mathfrak{L}}
\newcommand{\mfM}{\mathfrak{M}}	\newcommand{\mfN}{\mathfrak{N}}
\newcommand{\mfO}{\mathfrak{O}}	\newcommand{\mfP}{\mathfrak{P}}
\newcommand{\mfQ}{\mathfrak{Q}}	\newcommand{\mfR}{\mathfrak{R}}
\newcommand{\mfS}{\mathfrak{S}}	\newcommand{\mfT}{\mathfrak{T}}
\newcommand{\mfU}{\mathfrak{U}}	\newcommand{\mfV}{\mathfrak{V}}
\newcommand{\mfW}{\mathfrak{W}}	\newcommand{\mfX}{\mathfrak{X}}
\newcommand{\mfY}{\mathfrak{Y}}	\newcommand{\mfZ}{\mathfrak{Z}}
%Small Letters
\newcommand{\mfa}{\mathfrak{a}}	\newcommand{\mfb}{\mathfrak{b}}
\newcommand{\mfc}{\mathfrak{c}}	\newcommand{\mfd}{\mathfrak{d}}
\newcommand{\mfe}{\mathfrak{e}}	\newcommand{\mff}{\mathfrak{f}}
\newcommand{\mfg}{\mathfrak{g}}	\newcommand{\mfh}{\mathfrak{h}}
\newcommand{\mfi}{\mathfrak{i}}	\newcommand{\mfj}{\mathfrak{j}}
\newcommand{\mfk}{\mathfrak{k}}	\newcommand{\mfl}{\mathfrak{l}}
\newcommand{\mfm}{\mathfrak{m}}	\newcommand{\mfn}{\mathfrak{n}}
\newcommand{\mfo}{\mathfrak{o}}	\newcommand{\mfp}{\mathfrak{p}}
\newcommand{\mfq}{\mathfrak{q}}	\newcommand{\mfr}{\mathfrak{r}}
\newcommand{\mfs}{\mathfrak{s}}	\newcommand{\mft}{\mathfrak{t}}
\newcommand{\mfu}{\mathfrak{u}}	\newcommand{\mfv}{\mathfrak{v}}
\newcommand{\mfw}{\mathfrak{w}}	\newcommand{\mfx}{\mathfrak{x}}
\newcommand{\mfy}{\mathfrak{y}}	\newcommand{\mfz}{\mathfrak{z}}

\begin{document}

\setcounter{chapter}{4}

\chapter{Sobolev and interpolation inequalities}
\section{Interpolation inequalities}

\ex{}
{
    \begin{equation} \label{eq:1}
        \|u\|_{L^{2}}^{2} \leq \|u\|_{L^{2}} \|u'\|_{L^{2}} \text{ for } u \in C^{\infty}(\mathbb{R}) 
    \end{equation}
}

\begin{proof}
    Idea: use that \((u^2)' = 2uu'\) and Newton-Leibniz
    \begin{align*}
        u^2(x) &= 2 \int_{-\infty}^{x} u u' \,\mathrm{d}y = -2 \int_{x}^{\infty} u u' \,\mathrm{d}y \\
        &= \int_{-\infty}^{x} u u' \,\mathrm{d}y - \int_{x}^{\infty} u u' \,\mathrm{d}y \\
        & \leq \int_{-\infty}^{x} \vert u \vert  \vert u' \vert  \,\mathrm{d}y + \int_{x}^{\infty} \vert u \vert  \vert u' \vert \,\mathrm{d}y \\
        &= \int_{\mathbb{R}} \vert u \vert  \vert u' \vert \,\mathrm{d}y \\
        \text{(Hölder's inequality)} \quad & \leq \|u\|_{L^{2}} \|u'\|_{L^{2}}
    \end{align*}
\end{proof}

\qs{}
{
    Check that \ref{eq:1} is sharp. Namely, that \ref{eq:1} becomes equality for \(u(x) = e^{-\vert x \vert}\) (\(u(x)\) is an extremal function for \ref{eq:1}). Also, \ref{eq:1} is shift and scaling invariant, i.e. \(u_{\alpha}(x+h) = e^{-\alpha|x+h|}, h \in \mathbb{R}, \alpha>0\) -extremals.
}

\ex{Interpolation inequality}
{
    \(\Omega\)-domain in \(\mathbb{R}^{n}, u \in L_{p_1}(\Omega) \cap L_{p_2}(\Omega), 1 \leq p_1, p_2, < \infty, p_1 < p_2, \theta \in [0, 1], \frac{1}{p} = \frac{\theta}{p_1} + \frac{1-\theta}{p_2}\). Then
    \begin{equation}\label{eq:2}
        \|u\|_{L^{p}} \leq \|u\|_{L^{p_1}}^{\theta} \|u\|_{L^{p_2}}^{1 - \theta}
    \end{equation} 
}

\begin{proof}
    \[\int_{\mathbb{R}} \vert u \vert^p \,\mathrm{d}x = \int_{\mathbb{R}} \vert u \vert^{\theta p} \vert u \vert^{(1-\theta) p} \,\mathrm{d}x\]
    We apply Hölder's inequality with exponents \(P = \frac{p_1}{\theta p} \) and \(Q = \frac{p_2}{(1-\theta)p}\) (Note \(\frac{1}{P} + \frac{1}{Q} = \frac{\theta p}{p_1} + \frac{(1-\theta)p}{p_2} = 1\)). Then    
    \begin{align*}
        \int_{\mathbb{R}} \vert u \vert^{\theta p} \vert u \vert^{(1-\theta) p} \,\mathrm{d}x &\leq \left(\int_{\mathbb{R}} \vert u \vert^{p_1} \,\mathrm{d}x \right)^{\frac{1}{P}} \left(\int_{\mathbb{R}} \vert u \vert^{p_2} \,\mathrm{d}x \right)^{\frac{1}{Q}} \\
        &= \|u\|_{L^{p_1}}^{\theta} \|u\|_{L^{p_2}}^{1 - \theta}
    \end{align*}
\end{proof}

\section{Sobolev inequalities}
\ex{Sobolev inequality 1D}
{
    \(u \in C^{\infty}([0,1])\), want to prove the embedding \(W^{1, 1}([0,1]) \subset C([0,1])\), i.e.
    \begin{equation}\label{eq:3}
        \|u\|_{C([0,1])} \leq \|u\|_{L^{1}([0,1])} + \|u'\|_{L^{1}([0,1])}
    \end{equation}
}

\begin{proof}
    By the Newton-Leibniz formula, \(u(x) - u(y) = \int_{y}^{x} u'(s) \,\mathrm{d}s\). Also,
    \[|u(x)| \leq |u(y)| + \int_{0}^{1} |u'(s)| \,\mathrm{d}s \quad \forall x, y \in [0,1]\]
    By integration over \(y \in [0,1]\), 
    \[|u(x)| \leq \int_{0}^{1} |u(s)| \,\mathrm{d}s + \int_{0}^{1} |u'(s)| \,\mathrm{d}s = \|u\|_{W^{1, 1}([0, 1])}\]

    Taking supremum with respect to \(x \in [0,1]\), we obtain \(\|u\|_{C([0,1])} \leq \|u\|_{W^{1, 1}([0, 1])}\)
\end{proof}

\ex{Sobolev inequality 2D}
{
    \(u \in C^{\infty}([0,1]^2)\),  i.e. \(\Omega = [0,1]^2\), then \(W^{1, 1}(\Omega) \subset L^{2}(\Omega) : \|u\|_{L^{2}} \leq \|u\|_{W^{1, 1}(\Omega)}\)
}

\begin{proof}
    \(\int_{\Omega} u^{2}(x_1, x_2) \,\mathrm{d}x_1 \,\mathrm{d}x_2\) should be estimated. From \ref{eq:3}, we know that 
    \[\vert u(x_1, x_2) \vert \leq \int_{0}^{1} \vert u(s, x_2) \vert + |\partial_{x_1} u(s, x_2)| \,\mathrm{d}s \coloneqq f(x_2)\]
    \[\vert u(x_1, x_2) \vert \leq \int_{0}^{1} \vert u(x_1, s) \vert + |\partial_{x_2} u(x_1, s)| \,\mathrm{d}s \coloneqq g(x_1)\]
    Then 
    \begin{align*}
        \int_{\Omega} u^2 \,\mathrm{d}x &\leq \int_{0}^{1} g(x_1)f(x_2) \,\mathrm{d}x_1 \,\mathrm{d}x_2 \\
        &= \int_{0}^{1} f(x_2) \,\mathrm{d}x_2 \int_{0}^{1} g(x_1) \,\mathrm{d}x_1 \\
        &= \left(\int_{\Omega} \vert u(x_1, x_2) \vert + \vert \partial_{x_1} u(x_1, x_2) \vert \,\mathrm{d}x_1 \right) \left(\int_{\Omega} \vert u(x_1, x_2) \vert + \vert \partial_{x_2} u(x_1, x_2) \vert \,\mathrm{d}x_2 \right) \\
        &\leq \|u\|_{W^{1, 1}(\Omega)}
    \end{align*}
\end{proof}

\qs{Sobolev inequality 3D}
{
    \(u \in C^{\infty}(\bar{\Omega}), \Omega = (0,1)^3\). Prove that \(W^{1, 1}(\Omega) \subset L^{\frac{3}{2}}(\Omega)\), i.e.
    \begin{equation}\label{eq:4}
        \|u\|_{L^{\frac{3}{2}}(\Omega)} \leq \|u\|_{W^{1, 1}(\Omega)}
    \end{equation}
    Hint: first, prove that
    \[\int_{\Omega} f(x_1, x_2)g(x_2, x_3)h(x_1, x_3) \,\mathrm{d}x \leq \|f\|_{L^{2}} \|g\|_{L^{2}} \|h\|_{L^{2}}\]
    and use \ref{eq:3}.
}

\ex{}
{
    \(u \in C^{\infty}(\bar{\Omega}), \Omega = (0,1)^3\). Then 
    \begin{equation}
        \|u\|_{L^{6}(\Omega)} \leq C \|u\|_{W^{1, 2}(\Omega)}
    \end{equation}
}

\begin{proof}
    \begin{align*}
        \int_{\Omega} \vert u \vert^{6} \,\mathrm{d}x &= \int_{\Omega} (\vert u \vert^{4})^{\frac{3}{2}} \,\mathrm{d}x \\
        &\leq C \left(\int_{\Omega} \vert u \vert^{4} \,\mathrm{d}x + \int_{\Omega} u^{3}|\nabla u| \,\mathrm{d}x \right)^{\frac{3}{2}} \\
        (\text{by \eqref{eq:3}}) \quad &\leq C \left(\int_{\Omega} \vert u \vert^{4} \,\mathrm{d}x\right)^{\frac{3}{2}} + C \left(u^{3}|\nabla u| \,\mathrm{d}x \right)^{\frac{3}{2}} \\
        &\leq C \|u\|^{\frac{3}{2}\cdot \theta \cdot 4}_{L^{2}} \|u\|^{\frac{3}{2}\cdot (1-\theta) \cdot 4}_{L^{6}} + C \|u\|^{\frac{3}{2}\cdot 3}_{L^{6}} \|\nabla u\|^{\frac{3}{2}}_{L^{2}} \\
        \left(\theta = \frac{1}{4}\right) \quad &= C \|u\|^{\frac{3}{2}}_{L^{2}} \|u\|^{\frac{9}{2}}_{L^{6}} + C\|u\|^{\frac{9}{2}}_{L^{6}} \|\nabla u\|^{\frac{3}{2}}_{L^{2}} \\
        \left(\text{Young's inequality with } p=\frac{4}{5} \text{ and }q=-4\right) \quad &\leq \varepsilon \|u\|^{6}_{L^{6}} + C_{\varepsilon}(\|u\|_{L^{2}} + \|\nabla u\|_{L^{2}})^{6}
    \end{align*}
    Setting for example, \(\varepsilon = \frac{1}{2}\), we obtain
    \[\|u\|_{L^{6}(\Omega)} \leq C \|u\|_{W^{1, 2}(\Omega)}\]
\end{proof}

\thm{Sobolev embeddings}
{
    \begin{enumerate}[label=\bfseries\tiny\protect\circled{\small\arabic*}]
		\item \(W^{k_1, p_1}(\Omega) \subset W^{k_2, p_2}(\Omega) \Longleftrightarrow k_1 \geq k_2\) and \(1 \leq p_1, p_2 < \infty, k_1 - \frac{n}{p_1} \geq k_2 - \frac{n}{p_2}, \Omega \subset \mathbb{R}^{n}\).
		\item \(W^{k,p}(\Omega) \subset C^{\alpha}(\Omega)\) if \(\alpha < k - \frac{n}{p}\). If \(\alpha\) is not an integer, then the inequality is weak.
	\end{enumerate}
}

\ex{}
{
    \(H^{s}(\mathbb{R}^{n}) \subset C(\mathbb{R}^{n}) \iff s > \frac{n}{2} \) 
}

\begin{proof}
    \(u(x) = \int_{\mathbb{R}^{n}} e^{i\xi x} \hat{u}(\xi) \,\mathrm{d}\xi \)
    \begin{align*}
        \vert u(x) \vert &\leq \int_{\mathbb{R}^{n}} \vert \hat{u}(\xi) \vert \,\mathrm{d}\xi \\
        &= \int_{\mathbb{R}^{n}} \left(1+\vert \xi \vert ^{2} \right)^{-\frac{s}{2}} \left(1+\vert \xi \vert ^{2} \right)^{\frac{s}{2}} \vert \hat{u}(\xi) \vert \,\mathrm{d}\xi \\
        \text{(Hölder's inequality)} \quad &\leq \left(\int_{\mathbb{R}^{n}} \frac{1}{\left(1+\vert \xi \vert ^{2}  \right)^{s}} \,\mathrm{d}\xi \right)^{\frac{1}{2}} \left(\int_{\mathbb{R}^{n}} \left(1+\vert \xi \vert ^{2} \right)^{s} \vert \hat{u}(\xi) \vert^{2} \,\mathrm{d}\xi \right)^{\frac{1}{2}}
    \end{align*} 

    \(\int_{\mathbb{R}^{n}} \frac{1}{\left(1+\vert \xi \vert ^{2}  \right)^{s}} \,\mathrm{d}\xi < \infty \iff s>\frac{n}{2}\).
    Taking the supremum with respect to \(x \in \mathbb{R}^{n}\), we get
    \[\|u\|_{C(\mathbb{R}^{n})} \leq C_{s}\|u\|_{H^{s}(\mathbb{R}^{n})}\]   
\end{proof}

\thm{Interpolation inequalities}
{
    Let \(u \in W^{k_1, p_1}(\Omega) \bigcap W^{k_2, p_2}(\Omega), \theta \in [0,1], 1\leq p_1, p_2 \leq \infty\) with \(k = \theta k_1 + (1-\theta) k_2, \frac{1}{p} = \frac{\theta}{p_1} + \frac{1-\theta}{p_2} \). Then 
    \[
        \|u\|_{W^{k, p}} \leq C\|u\|_{W^{k_1, p_1}}^{\theta} \|u\|_{W^{k_2, p_2}}^{1-\theta}
    \]
}

\cor{Particular cases}
{
    \begin{enumerate}
        \item \(\|u\|_{H^{1}} \leq \|u\|_{L^{2}}^{\frac{1}{2}} \|u\|_{H^{2}}^{\frac{1}{2}}\)
        \item \(\|u\|_{L^{p}} \leq \|u\|_{L^{p}}^{\theta} \|u\|_{H^{2}}^{1-\theta}\) 
    \end{enumerate}
}

\section{Spaces with zero boundary traces}
\dfn{}
{
    \(W^{1, p}_{0}(\Omega) \coloneqq \left\{u \in W^{1, p}(\Omega), \left. u \right|_{\partial \Omega} = 0\right\} \)

    Equivalent definition: \(W^{1, p}_{0}(\Omega)\) = ``closure of \(C_{0}^{\infty}(\Omega)\) in \(W^{1, p}\) norm.''
}

\mlemma{}
{
    These two definitions are equivalent. \(u \in \text{``closure"} \colon u = \lim\limits_{n \to \infty} \varphi_{n}, \varphi_{n} \in C^{\infty}_{0}(\Omega) \implies \left. \varphi_{n} \right|_{\partial \Omega} = 0\). By continuity, \(\left. u \right|_{\partial \Omega} = 0\). The proof of the converse statement is more technical and is omitted.
}

\mprop{Friedrich's inequality}
{
    Let \(\Omega\) be a bounded domain and \(u \in W^{1, p}_{0}(\Omega)\). Then
    \begin{equation}\label{eq:5}
        \|u\|_{L^{p}} \leq C \|\nabla u\|_{L^{p}} 
    \end{equation}
}

\begin{proof}
    It is enough to prove \ref{eq:5} for \(\varphi \in C^{\infty}_{0}(\Omega)\). By the Newton-Leibniz formula,      
    \[    
        u(x_1, x') - u(-L, x') = u(x_1, x') = \int_{-L}^{x_1} \partial_{x_1}u(s, x') \,\mathrm{d}s 
    \] 

    \begin{align*}
        \vert u(x_1, x') \vert ^{p} &\leq \left(\int_{-L}^{L} \vert \partial_{x_1}u(s, x') \vert \,\mathrm{d}s \right)^{p} \\
        \text{(Hölder's inequality)} \quad & \leq C_{L} \int_{-L}^{L} \vert \partial_{x_1}u(s, x') \vert^{p} \,\mathrm{d}s
    \end{align*}
    Integration with respect to $x'$ gives us
    \[
        \int_{\mathbb{R}^{n-1}} \vert u(x_1, x') \vert ^{p} \,\mathrm{d}x' \leq C_{L}\|\partial_{x_1}u\|_{L^{p}}^{p}
    \]
    Finally, integrating over \(x_1 \in [-L, L]\), we obtain
    \[
        \|u\|_{L^{p}}^{p} \leq 2LC_{L} \|\partial_{x_1}u\|_{L^{p}}^{p}
    \]
\end{proof}

\cor{Equivalent norm in \(W^{1, p}_{0}(\Omega)\)}
{
    Homogeneous norm:
    \[
        \|u\|_{W^{1, p}_{0}(\Omega)} \coloneqq \|\nabla u\|_{L^{p}}
    \]
}

\begin{note}
    \(\left. u \right|_{\partial \Omega} = 0\) is important! Otherwise, \ref{eq:5} will fail for \(u \equiv c\). Since \(\nabla u\) defines \(u\) up to a constant; \(\left. u \right|_{\partial \Omega} = 0\) removes this constant.  
\end{note}

\mprop{Poincaré inequality}
{
    Let \(\Omega\) be a bounded domain with a smooth boundary and \(\left\langle u \right\rangle \coloneqq \frac{1}{\vert \Omega \vert } \int_{\Omega} u(x) \,\mathrm{d}x = 0\). Then 
    \[
        \|u\|_{L^{p}} \leq C \|\nabla u\|_{L^{p}}  
    \]   
}

\dfn{Sequential compactness}
{
    A metric space \((X, d)\) is compact if any sequence \(\{x_{n}\}_{n=1}^{\infty} \subset X\) has a convergent sub-sequence, i.e. there exists \(\{x_{n_{k}}\}_{k=1}^{\infty} \colon \lim\limits_{k \to \infty} x_{n_{k}} = x_0 \in X\)
}

\dfn{}
{
    A topological space \(X\) is compact if any covering of \(X\) by open sets has a finite sub-covering
}

\begin{note}
    In metric spaces, compactness is equivalent to sequential compactness.

    In general topological spaces, they are not related.
\end{note}

\thm{Hausdorff}
{
    Let \((X, d)\) be a metric space. Then \(X\) is compact \(\iff\) \(X\) is complete and totally bounded.
}

\dfn{}
{
    \(X\) is totally bounded if \(\forall \epsilon > 0, \exists\) covering of \(X\) by finitely many \(\epsilon\)-balls, i.e. \(X = \bigcup_{k=1}^{N} B_{\epsilon}(x_k), N = N(\epsilon)\) and \(\{x_k\}\) is an \(\epsilon\)-net in \(X\).  
}

\section{Why do we need compactness?}
Let \(X\) be compact and \(f \colon X \to Y \) be continuous, then \(f(X)\) is compact in Y. How do we solve PDEs of the form (or more general equations)?

\begin{equation}\label{eq:6}
    F(x) = 0
\end{equation}

\begin{enumerate}
    \item Construct approximate solutions
    
    \(F(x_n) = g_n\), where \(\lim\limits_{n \to \infty} g_n = 0\)
    \item Obtain a priori estimates, i.e. that \(\{x_n\}\)  is bounded in a proper space
    \item If \(\{x_n\}\) is pre-compact and \(F\) is continuous \(\implies x = \lim\limits_{x \to \infty} x_{n_{k}}\) is a solution of \ref{eq:6}.
\end{enumerate}

\thm{Arzelà-Ascoli}
{
    Let \(\Omega \subset \mathbb{R}^{n}\) be a bounded domain. Then \(V \subset C(\bar{\Omega})\) is compact iff:
    \begin{enumerate}
        \item \(V\)  is closed
        \item \(V\) is bounded
        \item \(V\) is equicontinuous \(= V\) has a common modulus of continuity
    \end{enumerate} 
}

\thm{Arzelà-Ascoli for \(L^p\)}
{
    Let \(\Omega \subset \mathbb{R}^{n}\) be a bounded domain, (and \(\partial \Omega\) smooth, although not needed), \(K \subset L^{p}(\Omega), 1 \leq p \leq \infty\). Then \(K\) is compact iff:
    \begin{enumerate}
        \item \(K\) is closed
        \item \(K\) is bounded
        \item \(K\) is equicontinuous in mean (possesses a joint modulus of continuity in \(L^p\)).
    \end{enumerate} 
}

\dfn{}
{
    Let \(f \in L^p(\Omega), 1 \leq p \leq \infty, \Omega \subset \mathbb{R}^{n}\) bounded (\(\partial \Omega\) smooth not needed). \(\omega \colon \mathbb{R}^{+} \to \mathbb{R}^{+}\) such that \(\lim\limits_{z \to 0} w(z) = 0\) is a modulus of continuity of \(f\) in \(L_{p}(\Omega)\) if 
    \[
        \int_{\Omega} \vert f(x+h) - f(x) \vert ^p \,\mathrm{d}x \leq \omega(\vert h \vert ), \quad \forall h \in \mathbb{R}^n,
    \]
    where we used to \(0\)-extension of \(f\) outside of \(\Omega\). 
}

\cor{}
{\label{cor:1}
    Let \(K = B_{1}(0) \in W^{1, p}(\Omega); \Omega \subset \mathbb{R}^{n}\) is bounded, \(\partial \Omega\) is smooth, \(1 \leq p < \infty\). Then \(K\) is pre-compact in \(L^{p}(\Omega)\).
}

\begin{proof}
    We need to check equicontinuity, i.e. estimate \(\int_{\Omega} \vert f(x+h) - f(x) \vert ^p \,\mathrm{d}x \).
    \[
        f(x+h) - f(x) = h \int_{0}^{1} \nabla f(x + sh) \,\mathrm{d}s  
    \]
    Taking modulus and \(p\)-th power of both sides, we get
    \[
        \vert f(x+h) - f(x) \vert ^p \leq \vert h \vert \int_{0}^{1} \vert \nabla f(x + sh) \vert ^p \,\mathrm{d}s 
    \]
    Finally, we take an integral over \(x \in \Omega\).
    \begin{align*}
        \int_{\Omega} \vert f(x+h) - f(x) \vert ^p \,\mathrm{d}x &\leq \vert h \vert \int_{0}^{1} \int_{\Omega}  \vert \nabla f(x + sh) \vert ^p \,\mathrm{d}x \,\mathrm{d}s \\
        &\leq C \vert h \vert
    \end{align*}
    \(\omega(z) = cz\) is a joint modulus of continuity.
\end{proof}

\dfn{}
{
    Let \(V \subset  W\) be Banach spaces. Then the embedding is compact if the unit ball of \(V\) is pre-compact in \(W\).
}

\begin{note}
    We proved that \(W^{1, p}(\Omega) \subset L^{p}(\Omega)\) is a compact embedding.
\end{note}

\cor{}
{
    \(W^{1, p}(\Omega) \subset L^{q}(\Omega)\) is a compact embedding if \(q < q^{*}\), where \(q^{*}\) is defined such that \(\frac{1}{q^{*}} = \frac{1}{p} - \frac{1}{n}\) and \(\Omega \subset  \mathbb{R}^{n}\), \(\Omega\) is bounded, \(\partial \Omega\) is smooth.
}

\begin{proof}
    Let us check equicontinuity. 
    \[
        \|f(\cdot + h) - f(\cdot)\|_{L^{q}} \leq \|f(\cdot + h) - f(\cdot)\|_{L^{p}}^{\theta} \|f(\cdot + h) - f(\cdot)\|_{L^{q^{*}}}^{1-\theta} 
    \]
    since \(p < q < q^{*}\) and \(0 < \theta < 1\). \(q^{*}\) is a critical exponent in Sobolev embeddings, indeed, \(W^{1, p}(\Omega) \subset L^{q}(\Omega) \implies 1-\frac{n}{p} \geq -\frac{1}{q}\). Then by corollary \ref{cor:1}, we have
    \begin{align*}
        \|f(\cdot + h) - f(\cdot)\|_{L^{p}}^{\theta} \|f(\cdot + h) - f(\cdot)\|_{L^{q^{*}}}^{1-\theta} &\leq C \vert h \vert ^{\theta}(2\|f\|_{L^{q^{*}}})^{1-\theta} \\
        &\leq C_1 \vert h \vert ^{\theta} \|f\|_{W^{1, p}}^{1-\theta} \\
        &\leq C_1 \vert h \vert ^{\theta}
    \end{align*}
\end{proof}

\noindent
General fact: \(W^{s_1, p_1}(\Omega) \subset W^{s_2, p_2}(\Omega)\), where \(\Omega\) is bounded, \(\partial \Omega\) is smooth. Embedding is compact \(\iff\) embedding is not critical.

\end{document}
\documentclass{report}

\input{preamble}
\input{macros}
\input{letterfonts}

\begin{document}

\setcounter{chapter}{4}
\chapter{Sobolev and interpolation inequalities}
\section{Interpolation inequalities}

\ex{}
{
    \begin{equation} \label{eq:1}
        \|u\|_{L^{2}}^{2} \leq \|u\|_{L^{2}} \|u'\|_{L^{2}} \text{ for } u \in \mathcal{C}^{\infty}(\mathbb{R}) 
    \end{equation}
}

\begin{proof}
    Idea: use that \((u^2)' = 2uu'\) and Newton-Leibniz
    \begin{align*}
        u^2(x) &= 2 \int_{-\infty}^{x} u u' \,\mathrm{d}y = -2 \int_{x}^{\infty} u u' \,\mathrm{d}y \\
        &= \int_{-\infty}^{x} u u' \,\mathrm{d}y - \int_{x}^{\infty} u u' \,\mathrm{d}y \\
        & \leq \int_{-\infty}^{x} \vert u \vert  \vert u' \vert  \,\mathrm{d}y + \int_{x}^{\infty} \vert u \vert  \vert u' \vert \,\mathrm{d}y \\
        &= \int_{\mathbb{R}} \vert u \vert  \vert u' \vert \,\mathrm{d}y \\
        \text{(Hölder's inequality)} \quad & \leq \|u\|_{L^{2}} \|u'\|_{L^{2}}
    \end{align*}
\end{proof}

\qs{}
{
    Check that \ref{eq:1} is sharp. Namely, that \ref{eq:1} becomes equality for \(u(x) = e^{-\vert x \vert}\) (\(u(x)\) is an extremal function for \ref{eq:1}). Also \ref{eq:1} is shift and scaling invariant, i.e. \(u_{\alpha}(x+h) = e^{-\alpha|x+h|}, h \in \mathbb{R}, \alpha>0\) -extremals.
}

\ex{Interpolation inequality}
{
    \(\Omega\)-domain in \(\mathbb{R}^{n}, u \in L_{p_1}(\Omega) \cap L_{p_2}(\Omega), 1 \leq p_1, p_2, \leq \infty, p_1 < p_2, \theta \in [0, 1], \frac{1}{p} = \frac{\theta}{p_1} + \frac{1-\theta}{p_2}\). Then
    \begin{equation}\label{eq:2}
        \|u\|_{L^{p}} \leq \|u\|_{L^{p_1}}^{\theta} \|u\|_{L^{p_2}}^{1 - \theta}
    \end{equation} 
}

\begin{proof}
    \[\int_{\mathbb{R}} \vert u \vert^p \,\mathrm{d}x = \int_{\mathbb{R}} \vert u \vert^{\theta p} \vert u \vert^{(1-\theta) p} \,\mathrm{d}x\]
    We apply Hölder's inequality with exponents \(P = \frac{p_1}{\theta p} \) and \(Q = \frac{p_2}{(1-\theta)p}\) (Note \(\frac{1}{P} + \frac{1}{Q} = \frac{\theta p}{p_1} + \frac{(1-\theta)p}{p_2} = 1\)). Then    
    \begin{align*}
        \int_{\mathbb{R}} \vert u \vert^{\theta p} \vert u \vert^{(1-\theta) p} \,\mathrm{d}x &\leq \left(\int_{\mathbb{R}} \vert u \vert^{p_1} \,\mathrm{d}x \right)^{\frac{1}{P}} \left(\int_{\mathbb{R}} \vert u \vert^{p_2} \,\mathrm{d}x \right)^{\frac{1}{Q}} \\
        &= \|u\|_{L^{p_1}}^{\theta} \|u\|_{L^{p_2}}^{1 - \theta}
    \end{align*}
\end{proof}

\section{Sobolev inequalities}
\ex{Sobolev inequality 1D}
{
    \(u \in \mathcal{C}^{\infty}([0,1])\), want to prove the embedding \(W^{1, 1}([0,1]) \subset \mathcal{C}([0,1])\), i.e.
    \begin{equation}\label{eq:3}
        \|u\|_{C([0,1])} \leq \|u\|_{L^{1}([0,1])} + \|u'\|_{L^{1}([0,1])}
    \end{equation}
}

\begin{proof}
    By the Newton-Leibniz formula, \(u(x) - u(y) = \int_{y}^{x} u'(s) \,\mathrm{d}s\). Also,
    \[|u(x)| \leq |u(y)| + \int_{0}^{1} |u'(s)| \,\mathrm{d}s \quad \forall x, y \in [0,1]\]
    By integration over \(y \in [0,1]\), 
    \[|u(x)| \leq \int_{0}^{1} |u(s)| \,\mathrm{d}s + \int_{0}^{1} |u'(s)| \,\mathrm{d}s = \|u\|_{W^{1, 1}([0, 1])}\]

    Taking supremum with respect to \(x \in [0,1]\), we obtain \(\|u\|_{C([0,1])} \leq \|u\|_{W^{1, 1}([0, 1])}\)
\end{proof}

\ex{Sobolev inequality 2D}
{
    \(u \in \mathcal{C}^{\infty}([0,1]^2)\),  i.e. \(\Omega = [0,1]^2\), then \(W^{1, 1}(\Omega) \subset L^{2}(\Omega) : \|u\|_{L^{2}} \leq \|u\|_{W^{1, 1}(\Omega)}\)
}

\begin{proof}
    \(\int_{\Omega} u^{2}(x_1, x_2) \,\mathrm{d}x_1 \,\mathrm{d}x_2\) should be estimated. From \ref{eq:3}, we know that 
    \[\vert u(x_1, x_2) \vert \leq \int_{0}^{1} \vert u(s, x_2) \vert + |\partial_{x_1} u(s, x_2)| \,\mathrm{d}s \coloneqq f(x_2)\]
    \[\vert u(x_1, x_2) \vert \leq \int_{0}^{1} \vert u(x_1, s) \vert + |\partial_{x_2} u(x_1, s)| \,\mathrm{d}s \coloneqq g(x_1)\]
    Then 
    \begin{align*}
        \int_{\Omega} u^2 \,\mathrm{d}x &\leq \int_{0}^{1} g(x_1)f(x_2) \,\mathrm{d}x_1 \,\mathrm{d}x_2 \\
        &= \int_{0}^{1} f(x_2) \,\mathrm{d}x_2 \int_{0}^{1} g(x_1) \,\mathrm{d}x_1 \\
        &= \left(\int_{\Omega} \vert u(x_1, x_2) \vert + \vert \partial_{x_1} u(x_1, x_2) \vert \,\mathrm{d}x_1 \right) \left(\int_{\Omega} \vert u(x_1, x_2) \vert + \vert \partial_{x_2} u(x_1, x_2) \vert \,\mathrm{d}x_2 \right) \\
        &\leq \|u\|_{W^{1, 1}(\Omega)}
    \end{align*}
\end{proof}

\qs{Sobolev inequality 3D}
{
    \(u \in \mathcal{C}^{\infty}(\bar{\Omega}), \Omega = (0,1)^3\). Prove that \(W^{1, 1}(\Omega) \subset L^{\frac{3}{2}}(\Omega)\), i.e.
    \begin{equation}\label{eq:4}
        \|u\|_{L^{\frac{3}{2}}(\Omega)} \leq \|u\|_{W^{1, 1}(\Omega)}
    \end{equation}
    Hint: first, prove that
    \[\int_{\Omega} f(x_1, x_2)g(x_2, x_3)h(x_1, x_3) \,\mathrm{d}x \leq \|f\|_{L^{2}} \|g\|_{L^{2}} \|h\|_{L^{2}}\]
    and use \ref{eq:3}.
}

\ex{}
{
    \(u \in \mathcal{C}^{\infty}(\bar{\Omega}), \Omega = (0,1)^3\). Then 
    \begin{equation}
        \|u\|_{L^{6}(\Omega)} \leq C \|u\|_{W^{1, 2}(\Omega)}
    \end{equation}
}

\begin{proof}
    \begin{align*}
        \int_{\Omega} \vert u \vert^{6} \,\mathrm{d}x &= \int_{\Omega} (\vert u \vert^{4})^{\frac{3}{2}} \,\mathrm{d}x \\
        &\leq C \left(\int_{\Omega} \vert u \vert^{4} \,\mathrm{d}x + \int_{\Omega} u^{3}|\nabla u| \,\mathrm{d}x \right)^{\frac{3}{2}} \\
        (\text{by \eqref{eq:3}}) \quad &\leq C \left(\int_{\Omega} \vert u \vert^{4} \,\mathrm{d}x\right)^{\frac{3}{2}} + C \left(u^{3}|\nabla u| \,\mathrm{d}x \right)^{\frac{3}{2}} \\
        &\leq C \|u\|^{\frac{3}{2}\cdot \theta \cdot 4}_{L^{2}} \|u\|^{\frac{3}{2}\cdot (1-\theta) \cdot 4}_{L^{6}} + C \|u\|^{\frac{3}{2}\cdot 3}_{L^{6}} \|\nabla u\|^{\frac{3}{2}}_{L^{2}} \\
        \left(\theta = \frac{1}{4}\right) \quad &= C \|u\|^{\frac{3}{2}}_{L^{2}} \|u\|^{\frac{9}{2}}_{L^{6}} + C\|u\|^{\frac{9}{2}}_{L^{6}} \|\nabla u\|^{\frac{3}{2}}_{L^{2}} \\
        \left(\text{Young's inequality with } p=\frac{4}{5} \text{ and }q=-4\right) \quad &\leq \varepsilon \|u\|^{6}_{L^{6}} + C_{\varepsilon}(\|u\|_{L^{2}} + \|\nabla u\|_{L^{2}})^{6}
    \end{align*}
    Setting for example, \(\varepsilon = \frac{1}{2}\), we obtain
    \[\|u\|_{L^{6}(\Omega)} \leq C \|u\|_{W^{1, 2}(\Omega)}\]
\end{proof}

\thm{Sobolev embeddings}
{
    \begin{enumerate}[label=\bfseries\tiny\protect\circled{\small\arabic*}]
		\item \(W^{k_1, p_1}(\Omega) \subset W^{k_2, p_2}(\Omega) \Longleftrightarrow k_1 \geq k_2\) and \(1 \leq p_1, p_2 < \infty, k_1 - \frac{n}{p_1} \geq k_2 - \frac{n}{p_2}, \Omega \subset \mathbb{R}^{n}\).
		\item \(W^{k,p}(\Omega) \subset C^{\alpha}(\Omega)\) if \(\alpha < k - \frac{n}{p}\).
	\end{enumerate}
}

\ex{}
{
    \(H^{s}(\mathbb{R}^{n}) \subset \mathcal{C}(\mathbb{R}^{n}) \iff s > \frac{n}{2} \) 
}

\begin{proof}
    \(u(x) = \int_{\mathbb{R}^{n}} e^{i\xi x} \hat{u}(\xi) \,\mathrm{d}\xi \)
    \begin{align*}
        \vert u(x) \vert &\leq \int_{\mathbb{R}^{n}} \vert \hat{u}(\xi) \vert \,\mathrm{d}\xi \\
        &= \int_{\mathbb{R}^{n}} \left(1+\vert \xi \vert ^{2} \right)^{-\frac{s}{2}} \left(1+\vert \xi \vert ^{2} \right)^{\frac{s}{2}} \vert \hat{u}(\xi) \vert \,\mathrm{d}\xi \\
        \text{(Hölder's inequality)} \quad &\leq \left(\int_{\mathbb{R}^{n}} \frac{1}{\left(1+\vert \xi \vert ^{2}  \right)^{s}} \,\mathrm{d}\xi \right)^{\frac{1}{2}} \left(\int_{\mathbb{R}^{n}} \left(1+\vert \xi \vert ^{2} \right)^{s} \vert \hat{u}(\xi) \vert^{2} \,\mathrm{d}\xi \right)^{\frac{1}{2}}
    \end{align*} 

    \(\int_{\mathbb{R}^{n}} \frac{1}{\left(1+\vert \xi \vert ^{2}  \right)^{s}} \,\mathrm{d}\xi < \infty \iff s>\frac{n}{2}\).
    Taking the supremum with respect to \(x \in \mathbb{R}^{n}\), we get
    \[\|u\|_{\mathcal{C}(\mathbb{R}^{n})} \leq C_{s}\|u\|_{H^{s}(\mathbb{R}^{n})}\]   
\end{proof}

\thm{Interpolation inequalities}
{
    Let \(u \in W^{k_1, p_1}(\Omega) \bigcap W^{k_2, p_2}(\Omega), \theta \in [0,1], 1\leq p_1, p_2 \leq \infty\) with \(k = \theta k_1 + (1-\theta) k_2, \frac{1}{p} = \frac{\theta}{p_1} + \frac{1-\theta}{p_2} \). Then 
    \[
        \|u\|_{W^{k, p}} \leq C\|u\|_{W^{k_1, p_1}}^{\theta} \|u\|_{W^{k_2, p_2}}^{1-\theta}
    \]
}

\cor{Particular cases}
{
    \begin{enumerate}
        \item \(\|u\|_{H^{1}} \leq \|u\|_{L^{2}}^{\frac{1}{2}} \|u\|_{H^{2}}^{\frac{1}{2}}\)
        \item \(\|u\|_{L^{p}} \leq \|u\|_{L^{p}}^{\theta} \|u\|_{H^{2}}^{1-\theta}\) 
    \end{enumerate}
}

\section{Spaces with zero boundary traces}
\dfn{}
{
    \(W^{1, p}_{0}(\Omega) \coloneqq \left\{u \in W^{1, p}(\Omega), \left. u \right|_{\partial \Omega} = 0\right\} \)

    Equivalent definition: \(W^{1, p}_{0}(\Omega)\) = ``closure of \(\mathcal{C}_{0}^{\infty}(\Omega)\) in \(W^{1, p}\) norm."
}

\mlemma{}
{
    These two definitions are equivalent. \(u \in \text{``closure"} \colon u = \lim\limits_{n \rightarrow \infty} \varphi_{n}, \varphi_{n} \in \mathcal{C}^{\infty}_{0}(\Omega) \implies \left. \varphi_{n} \right|_{\partial \Omega} = 0\). By continuity, \(\left. u \right|_{\partial \Omega} = 0\). The proof of the converse statement is more technical and is omitted.
}

\mprop{Friedrich's inequality}
{
    Let \(\Omega\) be a bounded domain and \(u \in W^{1, p}_{0}(\Omega)\). Then
    \begin{equation}\label{eq:5}
        \|u\|_{L^{p}} \leq C \|\nabla u\|_{L^{p}} 
    \end{equation}
}

\begin{proof}
    It is enough to prove \ref{eq:5} for \(\varphi \in \mathcal{C}^{\infty}_{0}(\Omega)\). By the Newton-Leibniz formula,      
    \[    
        u(x_1, x') - u(-L, x') = u(x_1, x') = \int_{-L}^{x_1} \partial_{x_1}u(s, x') \,\mathrm{d}s 
    \] 

    \begin{align*}
        \vert u(x_1, x') \vert ^{p} &\leq \left(\int_{-L}^{L} \vert \partial_{x_1}u(s, x') \vert \,\mathrm{d}s \right)^{p} \\
        \text{(Hölder's inequality)} \quad & \leq C_{L} \int_{-L}^{L} \vert \partial_{x_1}u(s, x') \vert^{p} \,\mathrm{d}s
    \end{align*}
    Integration with respect to $x'$ gives us
    \[
        \int_{\mathbb{R}^{n-1}} \vert u(x_1, x') \vert ^{p} \,\mathrm{d}x' \leq C_{L}\|\partial_{x_1}u\|_{L^{p}}^{p}
    \]
    Finally, integrating over \(x_1 \in [-L, L]\), we obtain
    \[
        \|u\|_{L^{p}}^{p} \leq 2LC_{L} \|\partial_{x_1}u\|_{L^{p}}^{p}
    \]
\end{proof}

\cor{Equivalent norm in \(W^{1, p}_{0}(\Omega)\)}
{
    Homogeneous norm:
    \[
        \|u\|_{W^{1, p}_{0}(\Omega)} \coloneqq \|\nabla u\|_{L^{p}}
    \]
}

\begin{note}
    \(\left. u \right|_{\partial \Omega} = 0\) is important! Otherwise \ref{eq:5} will fail for \(u \equiv c\). Since \(\nabla u\) defines \(u\) up to a constant; \(\left. u \right|_{\partial \Omega} = 0\) removes this constant.  
\end{note}

\mprop{Poincare inequality}
{
    Let \(\Omega\) be a bounded domain with a smooth boundary and \(\left\langle u \right\rangle \coloneqq \frac{1}{\vert \Omega \vert } \int_{\Omega} u(x) \,\mathrm{d}x = 0\). Then 
    \[
        \|u\|_{L^{p}} \leq C \|\nabla u\|_{L^{p}}  
    \]   
}

\end{document}